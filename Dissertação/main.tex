%% abtex2-modelo-trabalho-academico.tex, v-1.4 laurocesar
%% Copyright 2012-2013 by abnTeX2 group at http://abntex2.googlecode.com/ 
%%
%% This work may be distributed and/or modified under the
%% conditions of the LaTeX Project Public License, either version 1.3
%% of this license or (at your option) any later version.
%% The latest version of this license is in
%%   http://www.latex-project.org/lppl.txt
%% and version 1.3 or later is part of all distributions of LaTeX
%% version 2005/12/01 or later.
%%
%% This work has the LPPL maintenance status `maintained'.
%% 
%% The Current Maintainer of this work is the abnTeX2 team, led
%% by Lauro César Araujo. Further information are available on 
%% http://abntex2.googlecode.com/
%%
%% This work consists of the files abntex2-modelo-trabalho-academico.tex,
%% abntex2-modelo-include-comandos and abntex2-modelo-references.bib
%%

% ------------------------------------------------------------------------
% ------------------------------------------------------------------------
% abnTeX2: Modelo de Trabalho Acadêmico (tese de doutorado, dissertação de
% mestrado e trabalhos monográficos em geral) em conformidade com 
% ABNT NBR 14724:2011: Informação e documentação - Trabalhos acadêmicos -
% Apresentação
% ------------------------------------------------------------------------
% ------------------------------------------------------------------------

\documentclass[12pt,openright,twoside,a4paper]{abntex2}	% frente e verso
%\documentclass[article,11pt,oneside,a4paper]{abntex2}	
%\documentclass[12pt,oneside,a4paper]{abntex2}			% apenas frente

% ---
% PACOTES
% ---

% ---
% Pacotes fundamentais 
% ---

\usepackage{abntex2-ufc}
\usepackage{cmap}			% Mapear caracteres especiais no PDF
\usepackage{multirow}	
%\usepackage{fontspec}			% Pacote para utilizar OpenType fonts in Latex.
%\setmainfont{Times New Roman}		% Define a Fonte, mas precisa do "xelatex"~ or~ "lualatex".


\usepackage{lmodern}			% Usa a fonte Latin Modern
\usepackage[T1]{fontenc}		% Seleção de códigos de fonte.

\usepackage[utf8]{inputenc}		% Determina a codificação utilizada (conversão automática dos acentos)
\usepackage{makeidx}    	        % Cria o indice
\usepackage{hyperref}  			% Controla a formação do índice
\usepackage{lastpage}			% Usado pela Ficha catalográfica
\usepackage{indentfirst}		% Indenta o primeiro parágrafo de cada seção.
\usepackage{nomencl} 			% Lista de simbolos
\usepackage{color}			% Controle das cores


\usepackage{algpseudocode}
\usepackage{algorithm}
% sem todo
%\usepackage[disable]{todonotes}
% com todo
\usepackage{todonotes}
\usepackage[none]{hyphenat}
\usepackage{array}
\usepackage{float}
\restylefloat{table}

\usepackage{graphicx}			% Inclusão de gráficos
\usepackage{epstopdf}
\usepackage{subfig}

\DeclareGraphicsExtensions{.pdf,.png,.jpg} % Declara as extensões para Graphics

% pacotes para utilização de matemática
\usepackage{amsfonts}
\usepackage{amssymb}
\usepackage{amsthm}
\usepackage{amsmath}

%% pacotes para o cronograma
\usepackage{colortbl}
\usepackage{booktabs}
\definecolor{tcA}{gray}{0}


\usepackage{pdfpages}

% ---
	
% ---
% Pacotes de citações
% ---
%\usepackage[brazilian,hyperpageref]{backref} % Paginas com as citações na bibl
\usepackage[alf]{abntex2cite} % Citações padrão ABNT

% ==================== Definições de conveniência =================

\newcommand{\norma}[1]{\left|\left|#1\right|\right|}
\newtheorem{Def}{Definição}

% --- 
% CONFIGURAÇÕES DE PACOTES
% --- 

% ---
% Configurações do pacote backref
% Usado sem a opção hyperpageref de backref
%\renewcommand{\backrefpagesname}{Citado na(s) página(s):~}
% Texto padrão antes do número das páginas
%\renewcommand{\backref}{}
% Define os textos da citação
%\renewcommand*{\backrefalt}[4]{
%	\ifcase #1 %
%		Nenhuma citação no texto.%
%	\or
%		Citado na página #2.%
%	\else
%		Citado #1 vezes nas páginas #2.%
%	\fi}%
% ---

% ---
% Informações de dados para CAPA e FOLHA DE ROSTO
% ---
\titulo{Aplicação de aprendizado de máquina ao problema de evasão de discentes da UFC}
\autor{Abelardo Vieira Mota}
\local{Fortaleza}
\data{2015}
\orientador{João Paulo Pordeus Gomes}
%\coorientador{João Paulo Pordeus Gomes}
\instituicao{%
  Universidade Federal do Ceará
  \par
  Centro de Ciências 
  \par
  Departamento de Computação
  \par
  Programa de Pós-Graduação em Ciência da Computação}
\tipotrabalho{Dissertação (Mestrado)}
% O preambulo deve conter o tipo do trabalho, o objetivo, 
% o nome da instituição e a área de concentração 
%\preambulo{Modelo canônico de trabalho monográfico acadêmico em conformidade com as normas ABNT apresentado à comunidade de usuários \LaTeX.}
% ---
% ---
% Configurações de aparência do PDF final

% alterando o aspecto da cor azul
\definecolor{blue}{RGB}{41,5,195}

%\captionsetup{%
  %singlelinecheck=false,
  %tableposition=left
%}

% informações do PDF
\hypersetup{
     	%backref=true,
     	%pagebackref=true,
		%bookmarks=true,         		% show bookmarks bar?
		pdftitle={\imprimirtitulo}, 
		pdfauthor={\imprimirautor},
    	pdfsubject={\imprimirpreambulo},
		pdfkeywords={PALAVRAS}{CHAVES}{abnt}{abntex}{abntex2},
	    pdfproducer={LaTeX with abnTeX2}, 	% producer of the document
	    pdfcreator={\imprimirautor},
    	colorlinks=true,       		% false: boxed links; true: colored links
    	linkcolor=blue,          	% color of internal links
    	citecolor=blue,        		% color of links to bibliography
    	filecolor=magenta,      		% color of file links
		urlcolor=blue,
		bookmarksdepth=4
}
% --- 

% --- 
% Espaçamentos entre linhas e parágrafos 
% --- 

% O tamanho do parágrafo é dado por:
\setlength{\parindent}{1.3cm}

% Controle do espaçamento entre um parágrafo e outro:
\setlength{\parskip}{0.2cm}  % tente também \onelineskip



% Controles do espaçamento entre linhas:
%\OnehalfSpacing	% espaçamento um e meio (padrão); 
%\DoubleSpacing		% espaçamento duplo
\SingleSpacing		% espaçamento simples	
% --- 
	

% ---
% compila o indice
% ---
\makeindex
% ---

% ---
% Compila a lista de abreviaturas e siglas
% ---
% \makenomenclature
% ---

% ----
% Início do documento
% ----
\begin{document}


% ----------------------------------------------------------
% ELEMENTOS PRÉ-TEXTUAIS
% ----------------------------------------------------------

% ----------------------------------------------------------
% ELEMENTOS PRÉ-TEXTUAIS
% ----------------------------------------------------------
% \pretextual

% ---
% Capa
% ---
\imprimircapa

% ---
% Folha de rosto
% (o * indica que haverá a ficha bibliográfica)

\imprimirfolhaderosto

% ---
%\imprimirfolhaderosto*

% ---

% ---
% Inserir a ficha bibliografica
% ---

% Isto é um exemplo de Ficha Catalográfica, ou ``Dados internacionais de
% catalogação-na-publicação''. Você pode utilizar este modelo como referência. 
% Porém, provavelmente a biblioteca da sua universidade lhe fornecerá um PDF
% com a ficha catalográfica definitiva após a defesa do trabalho. Quando estiver
% com o documento, salve-o como PDF no diretório do seu projeto e substitua todo
% o conteúdo de implementação deste arquivo pelo comando abaixo:
%
% \begin{fichacatalografica}
%     \includepdf{fig_ficha_catalografica.pdf}
% \end{fichacatalografica}
%\begin{fichacatalografica}
%	\vspace*{\fill}					% Posição vertical
%	\hrule							% Linha horizontal
%	\begin{center}					% Minipage Centralizado
%	\begin{minipage}[c]{12.5cm}		% Largura
%	
%	\imprimirautor
%	
%	\hspace{0.5cm} \imprimirtitulo  / \imprimirautor. --
%	\imprimirlocal, \imprimirdata-
%	
%	\hspace{0.5cm} \pageref{LastPage} p. : il. (algumas color.) ; 30 cm.\\
%	
%	\hspace{0.5cm} \imprimirorientadorRotulo~\imprimirorientador\\
%	
%	\hspace{0.5cm}
%	\parbox[t]{\textwidth}{\imprimirtipotrabalho~--~\imprimirinstituicao,
%	\imprimirdata.}\\
%	
%	\hspace{0.5cm}
%		1. Palavra-chave1.
%		2. Palavra-chave2.
%		I. Orientador.
%		II. Universidade xxx.
%		III. Faculdade de xxx.
%		IV. Título\\ 			
%	
%	\hspace{8.75cm} CDU 02:141:005.7\\
%	
%	\end{minipage}
%	\end{center}
%	\hrule
%\end{fichacatalografica}

% ---


% ---
% Inserir folha de aprovação
% ---

% Isto é um exemplo de Folha de aprovação, elemento obrigatório da NBR
% 14724/2011 (seção 4.2.1.3). Você pode utilizar este modelo até a aprovação
% do trabalho. Após isso, substitua todo o conteúdo deste arquivo por uma
% imagem da página assinada pela banca com o comando abaixo:
%
% \includepdf{folhadeaprovacao_final.pdf}
%

%\begin{folhadeaprovacao}
%
%  \begin{center}
%    \vspace*{1cm}
%    {\ABNTEXchapterfont\large\imprimirautor}
%
%    \vspace*{\fill}\vspace*{\fill}
%    {\ABNTEXchapterfont\bfseries\Large\imprimirtitulo}
%    \vspace*{\fill}
%    
%    \hspace{.45\textwidth}
%    \begin{minipage}{.5\textwidth}
%        \imprimirpreambulo
%    \end{minipage}%
%    \vspace*{\fill}
%   \end{center}
    
%   Trabalho aprovado. \imprimirlocal, 24 de novembro de 2012:

%   \assinatura{\textbf{\imprimirorientador} \\ Orientador} 
%   \assinatura{\textbf{Professor} \\ Convidado 1}
%   \assinatura{\textbf{Professor} \\ Convidado 2}
   %\assinatura{\textbf{Professor} \\ Convidado 3}
   %\assinatura{\textbf{Professor} \\ Convidado 4}
      
%   \begin{center}
%    \vspace*{0.5cm}
%    {\large\imprimirlocal}
%    \par
%    {\large\imprimirdata}
%    \vspace*{1cm}
%  \end{center}
  
%\end{folhadeaprovacao}

% ---

% ---
% DEDICATÓRIA
% ---

% proposta sem dedicatória

%\begin{dedicatoria}
%   \vspace*{\fill}
%   \centering
%   \noindent
%   Dedico este trabalho.\vspace*{\fill}
%\end{dedicatoria}


% ---

% ---
% Agradecimentos
% ---
%\begin{agradecimentos}
%Agredicmentos..
%\end{agradecimentos}
% ---

% ---
% EPÍGRAFE
% ---

% proposta sem epígrafe

%\begin{epigrafe}
%    \vspace*{\fill}
%	\begin{flushright}
%		\textit{Essentially, all models are wrong, but some are useful.} \\
%		(George E. P. Box)
%	\end{flushright}
%\end{epigrafe}


% ---

% ---
% RESUMOS
% ---

% proposta sem resumos


% resumo em português
%\begin{abstract}

The ideal abstract will be brief, limited to one paragraph and no more than six ou seven sentences, to let readers scan it quickly for an overview of the paper's content.
 \vspace{\onelineskip}
    
 \noindent
 \textbf{Palavras-chaves}: Aprendizado de máquina. Evasão.
 
\end{abstract}

% resumo em inglês
%\begin{resumo}[Abstract]
%The ideal abstract will be brief, limited to one paragraph and no more than six ou seven sentences, to let readers scan it quickly for an overview of the paper's content.
% \vspace{\onelineskip}

% \noindent 
% \textbf{Key-words}: Machine Learning. Drop-out. Data Mining.
%\end{resumo}

% ---
% inserir lista de ilustrações
% ---
\pdfbookmark[0]{\listfigurename}{lof}
\listoffigures*
\cleardoublepage
% ---

% ---
% inserir lista de tabelas
% ---
\pdfbookmark[0]{\listtablename}{lot}
\listoftables*
\cleardoublepage
% ---

% ---
% inserir lista de abreviaturas e siglas
% A lista de Abreviaturas e Siglas pode ser facilmente montada com o pacote 
% nomencl. Abaixo segue um exemplo.
% ---
\nomenclature{Fig.}{Figura}
\nomenclature{$A_i$}{Area of the $i^{th}$ component} 
\nomenclature{456}{Isto é um número}
\nomenclature{123}{Isto é outro número}
\nomenclature{a}{primeira letra do alfabeto}
\nomenclature{lauro}{este é meu nome} 

\renewcommand{\nomname}{Lista de abreviaturas e siglas}
\pdfbookmark[0]{\nomname}{las}
\printnomenclature
\cleardoublepage
% ---

% ---
% inserir lista de símbolos
% ---
% O abnTeX2 não provê mecanismo para lista de símbolos.
% ---

% ---
% inserir o sumario
% ---
\pdfbookmark[0]{\contentsname}{toc}
\tableofcontents*
\cleardoublepage
% ---


% ----------------------------------------------------------
% ELEMENTOS TEXTUAIS
% ----------------------------------------------------------
% É possível usar \textual ou \mainmatter, que é a macro padrão do memoir.  

\mainmatter

\chapter{Introdução}

% [O problema]

O problema de evasão de discentes consiste no abandono, pelo discente, de um processo de estudos antes de sua conclusão. Essa definição pode ser detalhada especificando o escopo do processo de estudos: a evasão pode ser de um curso, de uma instituição de ensino, de cursos de uma determinada área, do sistema de ensino etc. O abandono do discente sem a conclusão do processo de estudos, considerando a finalidade desse processo, representa desperdício de recursos e de tempo de todos os envolvidos: discente, docentes, instituição de ensino e sociedade.

% todo: falar sobre estudos que apontam causas!

% [Justificativa]
O Programa de Apoio a Planos de Reestruturação e Expansão das Universidades Federais(REUNI), instituído pelo DECRETO Nº 6.096, DE 24 DE ABRIL DE 2007 \todo{como cita isso?}, possui como uma de suas diretrizes:

\begin{quote}
I - redução das taxas de evasão, ocupação de vagas ociosas e aumento de vagas de ingresso, especialmente no período noturno;
\end{quote}

\todo{sobre definições de evasão}

Na Universidade Federal do Ceará(UFC), de acordo com o Anuário estatístico \todo{quando sai o próximo?}de 2014, ano base 2013\todo{referenciar}, o indicador "Taxa de sucesso na graduação", definido como a proporção entre número de discentes diplomados e número de discentes ingressantes da graduação\todo{manual de indicadores do TCU, p.5}, esteve em 2013 com o menor valor desde 2008(Figura \ref{img:taxa-de-sucesso-ufc}). Já o indicador "Taxa de sucesso da graduação por curso", em 2013\todo{qual a definição?}, possuiu valor mínimo igual a 6.8\%, referente ao curso Ciências Sociais, habilitação em licenciatura(Tabela \ref{table:ts_2013}).

\begin{figure}[H]
	\includegraphics[scale=0.8]{img/taxa-de-sucesso-ufc.png}
	\caption{Taxa de Sucesso na Graduação - UFC}
	\label{img:taxa-de-sucesso-ufc}
\end{figure}

\begin{table}[H]
\begin{tabular}{llc}
\toprule
                            Curso &  Período & Taxa de Sucesso \\
\midrule
  Ciências Sociais - Licenciatura &  Noturno &            6.8\% \\
  Redes de Computadores - Quixadá &  Noturno &           13.3\% \\
          Geografia - Bacharelado &   Diurno &           15.3\% \\
        Letras - Português-Alemão &   Diurno &           17.6\% \\
           Engenharia Metalúrgica &   Diurno &           18.3\% \\
     Ciências Econômicas - Sobral &  Noturno &           20.9\% \\
 Sistemas de Informação - Quixadá &   Diurno &           22.0\% \\
          Filosofia - Bacharelado &  Noturno &           24.3\% \\
         Matemática - Bacharelado &   Diurno &           24.4\% \\
     Engenharia Elétrica - Sobral &   Diurno &           25.0\% \\
\bottomrule
\end{tabular}
\caption{Taxa de sucesso da graduação por curso na UFC em 2013 - 10 piores resultados, em ordem crescente}
\label{table:ts_2013}
\end{table}

Em DIRETRIZES GERAIS DO PROGRAMA DE APOIO A PLANOS DE REESTRUTURAÇÃO E EXPANSÃO DAS UNIVERSIDADES FEDERAIS REUNI p.4 \todo{referenciar}é ressaltado que o indicador "Taxa de conclusão dos cursos de graduação", cuja definição é igual à do "Taxa de Sucesso na graduação":

\begin{quote}
(...) não expressa diretamente as taxas de sucesso observadas nos cursos da universidade, ainda que haja uma relação estreita com fenômenos de retenção e evasão.
\end{quote}

Esse indicador não é sensível, por exemplo, à ocorrência de uma greve, fenômeno que causa atraso na formação dos discentes, podendo diminuir relevantemente a quantidade de diplomados em determinado ano. Funciona bem considerando o modelo em que, necessariamente, após a diplomação de uma turma de discentes, turma de igual tamanho irá ingressar. A realidade mostra que o processo de diplomação, iniciado com o ingresso do discente e finalizado com a emissão e recebimento de seu diploma, é mais complexo. Se adicionarmos o indivíduo que assume o papel de discente, a análise das causas da evasão torna-se mais complexa, trazendo à tona dimensões como a social e a econômica, além da vida pré universidade do indivíduo. Em \cite{unioeste}, por exemplo, informações sobre o trabalho do discente e se ele é casado foram os fatores mais relevantes na classificação dos discentes analisados com relação à evasão do curso. 

Uma das estratégias adotadas para diminuir as taxas de evasão é a identificação precoce de discentes com grande tendência para abandonarem seus cursos e a execução de ações que minimizem tal tendência. A identificação pode ser conduzida por observação do comportamento e resultados dos discentes, de forma subjetiva, pelos docentes e coordenadores de cursos, por exemplo. Em estudo realizado no departamento de engenharia elétrica da Eindhoven University of Technology\cite{Predicting_Students}, é relatado que em dezembro os discentes desse departamente recebem um aviso informando se são ou não aconselhados a continuarem no curso. Esse aviso é baseado na performance do discente no curso e em informações obtidas de professores do primeiro semestre e de discentes monitores. É relatado que o aviso parece ter bastante acurácia: geralmente discentes aconselhados a continuarem têm sucesso no próximo ano do curso, enquanto aqueles desaconselhados geralmente não continuam no curso.
Dois problemas decorrem dessa forma de identificação: sendo conduzida por pessoas, essa forma de identificação é limitada pelo conjunto de observações as quais o observador tem acesso; sendo subjetiva, seus resultados podem sofrer resistência para serem aceitos. A utilização de técnicas de aprendizado de máquina como forma de identificação pode contornar esses problemas, por, primeiro, fazer uso de dados registrados por sistemas de informação, provavelmente contendo informações mais amplas que as que uma pessoa pode observar; segundo, por fazer maior uso de dados registrados, sendo aceita mais facilmente como identificação objetiva. Nesse estudo foram utilizados diversos algoritmos de aprendizado de máquina com o objetivo de tentar detectar que um estudante irá abandonar seu curso. Foram utilizadas informações de discente referentes tanto ao período anterior ao seu ingresso na universidade, quanto ao posterior.

% [Objetivo]

Na UFC o estudo \cite{andriola} foi desenvolvido e objetivou analisar o problema da evasão de discentes através da análise da opinião de docentes e coordenadores sobre esse problema. O presente trabalho objetiva avaliar a aplicabilidade de técnicas de aprendizado de máquina ao problema de evasão de discentes na UFC a partir dos dados que seus sistemas de informação gerenciam. A UFC possui uma base de dados de informações sobre seus discentes gerada e mantida pelo sistema SIGAA(Sistema Integrado de Gestão de Atividades Acadêmicas). Para tanto é necessário que seja feito uma análise sobre a estrutura e qualidade dos dados disponíveis.

% [Definição de termos]

% [Importância]
\chapter{Aprendizado de máquina}

Aprendizado de máquina é uma subárea de Inteligência Artificial que agrupa conhecimentos sobre algoritmos e técnicas que permitam que um programa melhore sua performance a partir de dados. Mais formalmente, \cite[p.2, tradução nossa]{Tom_mitchell} define:

\begin{quote}
Um programa aprende a partir de uma experiência E, com relação a uma classe de tarefas T e a uma medida de performance P, se sua performance em tarefas da classe T, medida por P, melhora com a experiência E.
\end{quote}

% todo: adaptar para o problema de evasão?
Seja, por exemplo, o problema de identificação de autoria de textos. Uma das abordagens utilizada para resolvê-lo é verificar a similaridade entre o texto em análise e um conjunto de textos cujos autores já sejam conhecidos, denominado conjunto de treino, sendo reportado como o autor aquele cujos textos contidos no conjunto de treino sejam mais similares ao texto em análise. Nesse exemplo, o elemento experiência é representado por um conjunto de textos rotulados com seus respectivos autores; o elemento tarefa é a identificação do autor de um texto; o elemento medida de performance é a proporção de textos cujos autores são corretamente identificados.

\cite{Tom_mitchell}, para detalhar a modelagem de um programa com uma abordagem de aprendizado de máquina, apresenta uma sequência de passos para desenvolver um programa que aprenda a jogar xadrez, que aqui adaptamos para o problema de evasão de discentes.

% Escolha da medida de performance

O primeiro passo é a escolha da medida de performance. Um exemplo é a quantidade de discentes identificados que efetivamente abandonarem seus cursos. Uma questão surge: se o programa a ser desenvolvido for ser utilizado para identificar discentes que potencialmente abandonarão seus cursos, para os quais serão realizadas atividades para diminuição desse potencial, como contar precisamente quantos deles foram corretamente identificados? Como verificar que um discente que tinha a pretenção de abandonar o curso no momento da identificação pelo programa foi corretamente identificado se, após ações realizadas com a finalidade de diminuir seu potencial de evasão, ele concluiu o curso? A identificação pelo programa, seguido das ações, torna-se mais uma variável concorrente para o resultado do discente no curso. Outra questão relevante é o custo dos erros na identificação: considerando que identificar corretamente um discente com potencial de evasão implica em diminuição desse potencial, e que identificar incorretamente um discente sem potencial de evasão não implica em aumento desse potencial, concluímos que a performance do programa deve levar em consideração a diferença entre os custos de seus erros. 
% todo: evoluir sobre medidas de performance

% Escolha da experiência
% todo

% Escolha da função alvo/target function
% todo

% Escolha de uma representação para a função alvo
% todo

% Escolha de um algoritmo de aproximação 
% todo
\todo{algoritmos}
\todo{evaluating}

% Aplicação de aprendizado de máquina
% todo: melhorar

De acordo com \cite{Mitchell_discipline}, o conhecimento sobre aprendizado de máquina pode ser aplicado a diversas áreas, como, por exemplo, a reconhecimento de voz; a visão computacional, sendo utilizado no desenvolvimento de sistemas de reconhecimento facial; a controle de robôs.

\cite{ML_debt} 
\cite{ML_know}
\chapter{Evasão de discentes}

\section{Definição}

O fenômeno evasão de discente consiste na interrupção de um processo de aprendizado de um discente antes de sua conclusão. Ele pode ser caracterizado por um conjunto de atributos: o agente que interrompeu o processo, o discente, o escopo, o tempo cursado e o motivo.

A iniciativa de interromper o processo de aprendizado pode ter sido tanto do discente, nos casos de desistência e transferência, ou da instituição de ensino, nos casos de jubilamento.

Com relação ao atributo escopo, a evasão pode ser classificada em:

\begin{itemize}
\item Evasão de curso: refere-se à evasão de um curso.
\item Evasão de área de conhecimento: refere-se à evasão de um curso sem posterior ingresso em curso da mesma área de conhecimento.
\item Evasão de IES: refere-se à evasão de um curso sem posterior ingresso em curso da mesma instituição de ensino. 
\item Evasão do ensino: refere-se à evasão de um curso sem posterior ingresso em outro curso.
\end{itemize}

Com relação ao tempo cursado, \cite{evasao_panorama} indica que, em todo o mundo, a taxa de evasão no primeiro ano de curso é duas a três vezes maior que a dos anos seguintes.

\section{Dados de ocorrência}

A fim de quantificar a ocorrência de evasão de discentes, é necessária a utilização de métricas. A métrica que quantifica com maior exatidão o fenômeno, de acordo com \cite{evasao_panorama2} é a Acompanhamento de Coorte, consistindo na análise individualizada de cada discente. Para realizar tal análise são necessárias informações do histórico do discente durante o processo de aprendizado. Caso essas informações não estejam disponíveis, como quando a análise é feita sobre dados agregados, faz-se necessária a utilização de métricas de mais alto nível. Em \cite{mudanca_calculos} são definidas as métricas Taxa de Titulação e Evasão Anual.

% Taxa de sucesso

Definido com o nome Taxa de Sucesso em \cite{indicadores_TCU}, a Taxa de Titulação é a razão entre a quantidade de discentes diplomados em um ano e a quantidade de diplomados esperados para aquele ano, considerando o prazo esperado de conclusão(por exemplo, caso tenham ingressados 60 discentes em Computação em 2008 e diplomados 30 em 2012, considerando 4 anos o tempo esperado de conclusão do curso, o valor dessa métrica para o ano 2012 será 0.5).

\begin{equation}
Taxa\ de\ Sucesso=\frac{Nº\ de\ diplomados}{Nº\ de\ diplomados\ esperados}
\end{equation}

Este indicador pode gerar interpretações equivocadas por abstrair detalhes do fenômeno, como a ocorrência de ocupação, via transferência, de vagas ociosas, e por considerar apenas as conclusões no prazo esperado, podendo gerar dados inesperados, como percentuais acima de 100\%, indicando haver se formado no ano em questão mais discentes que o esperado, consequência, por exemplo, de discentes atrasados ou de discentes adiantados.

Em \ref{table:ts_2013} são apresentados os cinco maiores e cinco menores valores de Taxa de Sucesso para cursos da UFC no ano 2013, de acordo com \cite{anuario_estatistico}.

\begin{table}

\begin{tabular}{lll}
\toprule
                           Curso &  Período & Taxa de Sucesso \\
\midrule
 Ciências Sociais - Licenciatura &  Noturno &    6.8\% \\
 Redes de Computadores - Quixadá &  Noturno &   13.3\% \\
         Geografia - Bacharelado &   Diurno &   15.3\% \\
       Letras - Português-Alemão &   Diurno &   17.6\% \\
          Engenharia Metalúrgica &   Diurno &   18.3\% \\
  Educação Física - Licenciatura &   Diurno &  124.0\% \\
            Odontologia - Sobral &   Diurno &  135.0\% \\
                       Pedagogia &   Diurno &  143.6\% \\
        Geografia - Licenciatura &   Diurno &  219.0\% \\
        Filosofia - Licenciatura &  Noturno &  223.1\% \\
\bottomrule
\end{tabular}

\caption{Taxa de Sucesso na UFC, ano 2013 - 5 menores e 5 maiores resultados}
\label{table:ts_2013}

\end{table}

% Evasão anual

Definido em \cite{esclarecimentos_calculos} e \cite{mudanca_calculos}, é a razão entre a quantidade de rematrículas realizadas e a quantidade de rematrículas esperadas.
\begin{equation}
1 - \frac{M(n) - Ig(n)}{M(n-1) - Eg(n-1)}
\end{equation}
\begin{itemize}
\item $M(n)$: quantidade de matrículas na unidade de tempo $n$
\item $Eg(n)$: quantidade de egressos na undiade de tempo $n$
\item $Ig(n)$: quantidade de ingressantes na unidade de tempo $n$
\end{itemize}

Para derivar essa taxa, basta notarmos as seguintes equações e respectivos significados:
\begin{enumerate}
\item 
\begin{equation}
M(n) = E(n) + X(n) + Eg(n)
\end{equation}
significando que a quantidade de matrículas em determinada unidade de tempo $n$ é composta pelas quantidades de matrículas de discentes que evadirão em $n$, de discentes que persistirão, isto é, se matricularão em $n+1$, e de discentes que se diplomarão ao fim de $n$.
\item 
\begin{equation}
M(n) = X(n-1) + Ig(n)
\end{equation}
significando que a quantidade de matrículas em determinada unidade de tempo $n$ é composta pelas quantidades de matrículas de discentes que se matricularam em $n$ e em $n-1$, ou seja, que persistiram de $n-1$ a $n$, e de discentes que ingressaram em $n$.
\end{enumerate}

Fazendo as devidas manipulações nas equações, chegamos na equação de quantidade de evasões ao fim de determinada unidade de tempo:
\begin{equation}
E(n) = [M(n) - Eg(n)] - [M(n+1) - Ig(n+1)]
\end{equation}
Podendo ser interpretada como a diferença de quantidade máxima de discentes persistindo e a quantidade efetiva de discentes que persistiram.

A fórmula da evasão anual pode ser então definida como a proporção entre a quantidade de evasões em uma unidade de tempo e a quantidade máxima de discentes persistindo naquela unidade de tempo.

A definição de evasão anual até então analisada refere-se à evasão de cursos, de forma que um discente que tenha mudado de curso na mesma IES terá ao mesmo tempo contabilizados sua matrícula no período anterior($M(n-1)$) e como ingressante($Ig(n)$).

\todo{analisar o ciclo de vida de discente e derivar os indicadores}
Para a evasão anual de IES, é proposta a seguinte definição:
\begin{equation}
1 - \frac{M(n) - [Ig(n) - ITC(n)]}{M(n-1) - Eg(n-1)}
\end{equation}
\begin{itemize}
\item $ITC(n)$: quantidade de ingressantes no ano $n$ com forma de ingresso transferência interna
\end{itemize}

Já para a evasão anual do sistema, a ser aplicada a dados de várias, se possível todas, IES, a definição proposta é:
\begin{equation}
1 - \frac{M(n) - [Ig(n) - ITC(n) - ITIES(n)]}{M(n-1) - Eg(n-1)}
\end{equation}
\begin{itemize}
\item $ITIES(n)$: quantidade de ingressantes no ano $n$ com forma de ingresso transferência externa
\end{itemize}

\todo[inline]{apresentar dados do brasil}
\todo[inline]{apresentar dados da UFC}

\section{Consequências}

Sob a perspectiva de que o processo de aprendizado é um investimento e que o resultado esperado é a sua conclusão, o fenômeno evasão de discente pode ser considerado um problema: para a sociedade, com a frustração da expectativa de formação de profissionais e pesquisadores qualificados; para a instituição de ensino, caso tenha realizado investimentos em infraestrutura e em recursos humanos para atender a uma quantidade esperada de discentes ativos maior que a real, ocorrendo desperdício de recursos; caso seu orçamento seja ou função da quantidade de discentes ativos, no caso das instituições de ensino particulares, ou função da quantidade de discentes diplomados, no caso das instituições de ensino superior públicas\todo{validar o impacto da evasão no orçamento das IFES}; para o indivíduo que investiu tempo, dinheiro e dedicação, mas não terá os benefícios da conclusão da graduação, crítica no caso das profissões que exigem, para serem exercidas, diploma de graduação.\todo{copiado da introdução - na introdução, resumir}

Em \cite{evasao_global} foi estimado um prejuízo econômico, decorrente da evasão de discentes de graduação no período de 2007 a 2012, para a UFPB, de R\$ 415.032.704,52. A estimativa considera perdidos os recursos financeiros investidos para manutenção do discente que não concluiu a graduação, utilizando a fórmula:

$Perda\ Anual=n\_evadidos \times t\_permanencia \times v\_aluno$

onde:
\begin{itemize}
\item $n\_evadidos$ representa a quantidade, por ano, média de discentes que evadiram no período, considerada a média aritmética do total de discentes que evadiram no período pela quantidade de anos do período.
\item $t\_permanencia$ representa o tempo de permanência esperado de um discente antes de evadir.
\item $v\_aluno$ representa o custo corrente com hospital universitário por aluno corrente, indicador definido pelo TCU\cite{indicadores_TCU}.
\end{itemize}

Aplicando a fórmula para os dados da UFC, obtemos o resultado \todo{estimar o prejuízo econômico da evasão para a UFC}.

\section{Causas}

\cite{evasao_panorama2} indica um conjunto de causas mais comuns de evasão do sistema de ensino superior no Brasil, dentre elas:

\begin{enumerate}

\item Baixa qualidade da educação básica brasileira.
% mensurar o nível de conhecimento do ingressante

\item Limitação das políticas de financiamento ao estudante.
% mensurar a ocorrência de abandono por necessidade financeira

\item Escolha precoce da especialidade profissional.
% mensurar o nível de especialização dos cursos, rigidez para mudança de curso

\item Dificuldade de mobilidade estudantil, indicada como a dificuldade de transferência entre IES e de aproveitamento de créditos.
% mensurável pelos processos de transferência/aproveitamento de créditos

\item Falta de pressão para combater a evasão.
% mensurável pelas ações adotadas pela instituição

\item Enorme quantidade de docentes despreparados para o ensino e para lidar com o aluno real.
% mensurável por atributos do docente(lattes?)

\end{enumerate}

No trabalho \cite{andriola} são apresentados os resultados de pesquisa feita com uma amostra(tamanho 86 de um universo de tamanho 412) de discentes evadidos no período de 1999 e 2000 acerca dos motivos que os levaram a abandonar os cursos. Os seguintes motivos foram apresentados:

\begin{enumerate}

\item Incompatibilidade entre horários de trabalho e de estudo(destacado por 39.4\% dos evadidos).

\item Aspectos familiares(por exemplo: necessidade de dedicar-se aos filhos menores) e desmotivação com os estudos(justificado por 20\% dos evadidos).

\item Precariedade das condições físicas do curso ou inadequação curricular(mencionado por 10\% dos evadidos).

\end{enumerate}

\section{Soluções}

Em \cite{evasao_panorama2} são apresentados "Sete pontos para baixar a evasão":

\begin{enumerate}

\item Estabelecer um grupo de trabalho encarregado de reduzir a evasão.
% necessários dados - conhecer o fenômeno para combatê-lo

\item Avaliar as estatísticas da evasão.
% não só dados brutos, mas análises, visualizações, clusterizações etc

\item Determinar as causas da evasão.
% análise de fatores

\item Estimular a visão da IES centrada no aluno.

\item Criar condições que atendam aos objetivos que atraíram os alunos.
% é necessário conhecer o que atraiu os alunos

\item Tornar o ambiente e o trânsito na IES agradáveis aos alunos.

\item Criar programa de aconselhamento e orientação dos alunos.
% predição de discentes que precisam/se beneficiarão de aconselhamento/orientação.

\end{enumerate}

\section{Aplicação de aprendizado de máquina}

\todo[inline]{falar das oportunidades, diante de análises de causa e das soluções}

\subsection*{Trabalhos relacionados}

% trabalhos

% Predicting Students Drop Out: A Case Study \cite{Predicting_Students}

Em \cite{Predicting_Students} são aplicados algoritmos de aprendizado de máquina a dados de discentes do Electrical Engineering department, Eindhoven University of Technology, considerando o período de 2000 a 2009, com o objetivo de identificar discentes em grupos de risco de evasão. É relatado que esse departamento já avaliava os discentes com relação ao risco de evasão, mas de forma subjetiva. Os dados dos discentes são particionados em pré universidade e pós universidade, gerando três bases de dados de treinamento, a primeira consistindo nos dados pré universidade, a segunda nos dados pós universidade e a terceira com todos os dados. São utilizados os algoritmos OneRule, CART, C4.5, BayesNet, SimpleLogistic, JRip e Random Forest. É apresentado um resultado de 68\% de acurácia com o algoritmo OneRule aplicado à primeira base de dados, sem diferença significante na performance dos demais algoritmos. O mesmo resultado repete-se com as demais bases, mudando apenas o valor da acurácia alcançada, sendo igual a 76\% para a segunda base de dados e 75\% para a terceira. O estudo ressalta o maior custo da ocorrência de falsos negativos que de falsos positivos na identificação de discentes com risco de evasão. Ocorre que, argumenta-se, há prejuízo maior em não oferecer apoio a um discente com risco de evasão do que oferecer, desnecessariamente, apoio a um discente sem tal risco. O estudo faz uso então de uma matriz de custo, com o algoritmo CostSensitiveClassifier, obtendo melhores diminuição na ocorrência de falsos negativos, mas com perdas de acurácia. 

% Evaluating performance and dropouts of undergraduates using educational data mining \cite{EDM_ufrj}

Em \cite{EDM_ufrj} são aplicados algoritmos de aprendizado de máquina a dados de discentes de seis cursos da Universidade Federal do Rio de Janeiro(UFRJ), com o objetivo de identificar discentes que não terão pelo menos uma aprovação no segundo semestre de seus cursos. Os cursos considerados foram: Direito, Farmácia, Física, Engenharia Civil, Engenharia Mecânica, Engenharia de Produção. É indicado que esses cursos foram escolhidos por pertencerem a departamentos distintos, com perfis de discentes distintos. É observado também que tais cursos diferem com relação à quantidade de discentes ingressantes, à taxa de evasão registrada e à efetividade de certas práticas de ensino. Para cada curso são desenvolvidas uma base de dados de treinamento e uma base de dados de teste, composta pelos dados de seus discentes de primeiro semestre. Para a base de dados de treinamento foram utilizados os dados dos anos pares(de 1994.1 a 2008.1), já para a de teste foram utilizados os dados dos anos ímpares(de 1995.1 a 2009.1). O algoritmo de aprendizado utilizado foi o Naïve Bayes. São apresentados os resultados utilizando as medidas acurácia, taxa de verdadeiros positivos, taxa de verdadeiros negativos e Kappa\todo[inline]{citar a coefficient for agreement for nominal scale}. A acurácia, por exemplo, varia de 70\% a 100\% entre as bases nas quais o modelo desenvolvido foi testado.

Os estudos analisados fizeram uso de apenas uma definição de evasão, a evasão do curso, utilizando experiência de feedback indireto. O uso de experiência de feedback direto, considerando a definição de evasão no curso, torna mais complexa a coleta de dados para treinamento: considerando que um curso possa ser concluído com uma duração máxima de 10 anos, por exemplo, apenas após 10 anos do ingresso de um discente é que seus dados poderão ser utilizados. Outros fatores podem afetar esse prazo, como o trancamento do curso, a ocorrência de greves etc.

Considero que os estudos analisados não realizaram um estudo mais criterioso sobre o problema em questão, focando os esforços mais na utilização de algoritmos de aprendizado de máquina que na análise do problema, de suas diversas definições, dos atributos utilizados, de como utilizar os resultados obtidos para diminuir o problema etc.

% Educational Data Mining with Focus on Dropout Rates 
\cite{EDM_dropout_rates}
\todo[inline]{Educational Data Mining with Focus on Dropout Rates}

% Identificação dos Fatores que Influenciam a Evasão em Cursos de Graduação 
\cite{EDM_ufrj2}
\todo[inline]{Identificação dos Fatores que Influenciam a Evasão em Cursos de Graduação}

% Analysis of Student Data for Retention Using Data Mining Techniques 
\cite{EDM_retention}
\todo[inline]{Analysis of Student Data for Retention Using Data Mining Techniques}

% Taxonomia de features
\subsection*{Taxonomia de features}

% Tabela com estudos x features

\subsection*{Modelos de aprendizado de máquina}

% Tabela com estudos x modelos

\subsection*{Avaliação dos resultados}

% Tabela com estudos x avaliação

\subsection*{Ferramentas}

% Tabela com estudos x ferramentas

\subsection*{Resultados}

% Tabela com estudos x resultados

\subsection*{Conclusões}

% Tabela com estudos x conclusões
\chapter{Pontos de partida}

Como pontos de partida, consideraremos a análise de dados para avaliação de um conjunto de assunções e o desenvolvimento de modelos de aprendizado de máquina úteis no entendimento do fenômeno e combate de seus problemas.

\subsection*{Assunções a serem avaliadas}

As assunções a serem avaliadas são listadas indicando em que são baseadas e como serão avaliadas:

\begin{enumerate}

\item 

	\begin{itemize}

		\item As informações disponíveis ao fim do primeiro ano de um curso permitem desenvolver modelos de predição de evasão de discentes no primeiro ano.
		\item \textbf{Baseado em}: a quantidade de discentes que evadem no primeiro ano ser maior que a quantidade de discentes que evadem nos anos subsequentes. \todo{formular melhor}
		\item \textbf{Como será avaliada}: serão desenvolvidos e avaliados com dados de ingressantes de 2015 modelos de predição de evasão no primeiro ano.

	\end{itemize}
	
\item 

	\begin{itemize}

		\item O desempenho de um discente no ENEM em uma área de conhecimento permite predizer seu desempenho em disciplinas dessa mesma área ou de área similar.
		\item \textbf{Baseado em}: a ideia de que os conhecimentos testados no ENEM são pré requisitos para o aprendizado dos conhecimentos do curso.
		\item \textbf{Como será avaliada}: serão desenvolvidos e avaliados com dados de ingressantes de 2015 modelos de predição, a partir do desempenho no ENEM, do desempenho em disciplinas do primeiro ano do curso.

	\end{itemize}
	
\item 

	\begin{itemize}

		\item Conhecer as causas da evasão não implica em conhecer as causas da persistência.
		\item \textbf{Baseado em}: \cite{Tinto_completing} afirma que, apesar de a evasão e a persistência serem fenômenos relacionados, o entendimento das causas da evasão não implica, necessariamente, no entendimento das causas da persistência.
		\item \textbf{Como será avaliada}: \todo{analisar melhor. a ideia aqui é de dois fenômenos aparentemente um contrário ao outro, onde a remoção das causas de um não implica na ocorrência do outro}

	\end{itemize}
	
\item 

	\begin{itemize}

		\item Atributos do currículo de um curso, proporção de componentes curriculares obrigatórios, sinergia entre disciplinas obrigatórios de cada semestre, distribuição de disciplinas teóricas e práticas, distribuição de turmas nos turnos do dia estão correlacionados com a taxa de evasão anual do curso.
		\item \textbf{Baseado em}: a intuição de que quanto maior a proporção de componentes curriculares obrigatórios, menor a liberdade de escolha, pelo discente, de sua formação; de que é mais difícil o aprendizado conteúdos relacionados que não relacionados; de que quão mais distantes forem cursadas disciplinas teóricas e práticas, de mesma conteúdo ou de conteúdos relacionados, mais difícil será o aprendizado; de que quanto mais distribuído nos turnas as turmas de disciplinas obrigatórias de um semestre de um currículo forem, menor a liberdade, do discente, de realizar outras atividades, como estágio.
		\item \textbf{Como será avaliada}: serão desenvolvidos e avaliados modelos de predição da taxa de evasão anual de um curso a partir de dados de seus currículos.

	\end{itemize}
	
\item 

	\begin{itemize}

		\item Quão mais distante forem o histórico de um discente e o currículo ao qual ele está vinculado, maior a probabilidade de ele abandonar o curso.
		\item \textbf{Baseado em}: a intuição de que desvios do currículo são indícios de problemas para o discente, como é o caso de reprovação em uma disciplina obrigatória.
		\item \textbf{Como será avaliada}: serão desenvolvidas e avaliadas métricas para mensurar a distância entre o histórico de um discente e o currículo ao qual ele está vinculado.

	\end{itemize}
	
\item

	\begin{itemize}
	
		\item A taxa de evasão anual no primeiro ano de curso é maior para discentes que ingressaram via 2º opção do Sisu que para aqueles que ingressaram via 1º opção do Sisu.
		\item \textbf{Baseado em}: o fato de que o discente que ocupa vaga de 2º opção continua no processo de seleção do Sisu, concorrendo por vagas com sua inscrição de 1ª opção.
		\item \textbf{Como será avaliada}: serão calculadas as taxas de evasão anual para discentes ingressantes via Sisu via 1ª opção e via 2ª opção.
	\end{itemize}

\end{enumerate}

\subsection*{Modelos de aprendizado de máquina a serem desenvolvidos}

Os modelos de aprendizado de máquina a serem desenvolvidos são:
\todo[inline]{indicar como serão utilizados para alcançar os business goals}

\begin{enumerate}

\item Modelo de predição de evasão por um discente no primeiro ano a partir de informações de seu desempenho em disciplinas obrigatórias do 1º semestre.

\item Modelo de predição de retenção\todo{definir no capítulo sobre evasão} de um discente a partir de informações de seu desempenho em disciplinas obrigatórias do 1º semestre.

\item Modelo de predição do desempenho de um discente em disciplinas obrigatórias do primeiro ano a partir do seu desempenho no ENEM.

\item Modelo de predição do desempenho de um discente em disciplinas a partir de seu desempenho em disciplinas pré requisito.

\end{enumerate}


\chapter{Metodologia}

\todo[inline]{O capítulo todo é um resumo do CRISP-DM 1.0... fico dando \cite em todo parágrafo?}

\section{Processo CRISP-DM 1.0}

% Por que seguir um processo?
\todo[inline]{Por que um processo?}

% Reproducibilidade/padronização/etc
\cite{ML_know} \cite{balance-anarchy} \cite{ML_debt} \cite{replicability}

% Por que CRISP-DM?
\todo[inline]{Por que CRISP-DM?}
\cite{CRISP-DM-KDD-SEMMA}

% História do CRISP-DM
%% p. 1-2
\todo[inline]{História do CRISP-DM}
CRISP-DM(CRoss-Industry Standard Process for Data Mining) é um processo de aplicação de Mineração de Dados, desenvolvido pelo CRISP-DM Special Interest Group e publicado em 2000. Foi concebido em 1996 por três empresas que utilizavam Mineração de Dados: DaimlerChrysler(à época Daimler-Benz), SPSS(à época ISL) e NCR; motivadas pela incerteza com relação à qualidade de seus trabalhos, pelo questionamento de se toda nova empresa que quiser aplicar Mineração de Dados terá que passar pelo aprendizado que passaram, baseado em tentativa e erro, e como garantir, para seus clientes, que Mineração de Dados era uma área suficientemente madura para ser incorporada a seus processos de negócio. Em 1999 foi publicado um draft do CRISP-DM versão 1.0, sendo aplicado pela DaimlerChrysler, SPSS e NCR a vários tipos de aplicações, indústrias e problemas de negócio, sendo considerado, então, validado suficientemente para ser publicado e distribuído\cite{CRISP-DM}.
\todo[inline]{E hoje em dia?}

% Overview do CRISP-DM
%% p. 6
\todo[inline]{figura com diagrama}
\todo[inline]{Overview}
CRISP-DM segue uma estrutura hierárquica, composta de quatro níveis de abstração(do mais genérico ao mais específico):\todo{traduzir ou não?} fase, tarefa genérica, tarefa especializada e instância de processo. Os dois primeiros níveis, fase e tarefa genérico, foram modelados a fim de serem genéricos o suficiente para atenderem às todas aplicações de Mineração de Dados; completos, abrangendo todo o processo de Mineração de Dados; e estáveis, sendo aplicáveis tanto para as técnicas de Mineração de Dados existentes, quanto às que venham a ser desenvolvidas. O terceiro nível, tarefa especializada, é composto pelas tarefas a serem executadas em situações específicas para alcançar os objetivos das tarefas genéricas. Exemplificando, seja a tarefa genérica Limpar dados: a ela relacionadas estão as tarefas especializadas Limpar dados numéricos e Limpar dados categóricos. O quarto nível, instância de processo, é composto pelos registros de ações, decisões e resultados de uma execução do processo.\cite{CRISP-DM}

% Sobre não ser linear, mas interativo incremental
Apesar de a representação do processo sugerir que ele é composto por uma sequência fixa de fases, na prática as tarefas podem ser executadas seguindo outras ordens: é o caso de, por exemplo, na tarefa Avaliação do modelo ser verificado que são necessários mais dados, a serem adquiridos através de tarefas que, de acordo com o diagrama, já foram executadas.

% Mapeamento modelo -> instância
%% p. 7-8
Para o mapeamento do modelo em uma instância do processo, a especificação do CRISP-DM 1.0 identifica como relevantes quatro dimensões do contexto de Mineração de Dados: domínio de aplicação, tipo de problema de Mineração de Dados, aspectos técnicos e ferramentas e técnicas. Os valores dessas dimensões são utilizados nas decisões sobre que tarefas específicas podem ou devem ser executadas.

O processo de mapeamento do CRISP-DM a uma instância do processo é, de acordo com o CRISP-DM 1.0, composto pelas etapas:
\begin{enumerate}
\item Analisar o contexto, identificando os valores para as dimensões domínio de aplicação, tipo de problema de Mineração de Dados, aspectos técnicos e ferramentas e técnicas
\item Remover do modelo CRISP-DM os detalhes não aplicáveis ao contexto analisado
\item Adicionar detalhes específicos do contexto analisado
\item Especializar(ou instanciar) elementos genéricos do modelo de acordo com características concretas do contexto
\item Possivelmente renomear elementos genéricos do modelo a fim de tornar mais explícito seu significado, de acordo com o contexto
\end{enumerate}

\todo[inline]{tabela com exemplos de valores para as dimensões -> p.7}

\todo[inline]{figura com representação do processo}
\begin{figure}[H]
	\includegraphics[scale=0.8]{img/CRISP-DM-main.png}
	\caption{Fases do CRISP-DM}
	\label{img:CRISP-DM-diagram}
\end{figure}

% Overview das fases 
%% p. 10-11
A seguir segue uma descrição breve de cada uma das fases:

\begin{enumerate}

\item \textbf{Análise do negócio}:
O objetivo desta fase é entender os requisitos e objetivos do projeto sob uma perspectiva de negócio, traduzí-los para requisitos e objetivos sob uma perspectiva de Mineração de Dados e então traçar um plano preliminar para alcançá-los.

\item \textbf{Análise dos dados}:
Esta fase inicia com uma coleta inicial de dados e segue para o estudo dos dados a fim de identificar problemas de qualidade, obter insights e detectar possíveis subconjuntos de dados que permitam desenvolver hipóteses sobre informações que não estejam presentes.

\item \textbf{Preparação dos dados}:
Esta fase é composta por atividades necessárias para gerar, a partir dos dados inicialmente coletados, um conjunto de dados a ser utilizado pelas ferramentas de modelagem. Inclui atividades como seleção de tabelas, registros, atributos, transformações e limpeza de dados.

\item \textbf{Modelagem}:
Nesta fase são aplicadas técnicas de modelagem e os modelos desenvolvidos são otimizados. Normalmente aplicam-se ao problema mais de uma técnicas de modelagem. Como algumas técnicas de modelagem podem exigir que os dados estejam em dado formato, pode ser necessário voltar para a fase Preparação dos dados.

\item \textbf{Avaliação}:
Esta fase é iniciada quando já foi desenvolvido um modelo com alta qualidade, do ponto de vista da Mineração de Dados. Nela são avaliados a adequação do modelo como ferramenta para alcançar o objetivo de negócio que motivou o projeto e a qualidade da instância do processo. Ela termina com a decisão pela utilização ou não dos resultados obtidos.

\item \textbf{Instalação}:\todo{qual a tradução para deployment?}
Após o desenvolvimento de um modelo, faz-se necessário que ele seja disponibilizado para os usuários finais, seja na forma de relatórios, seja na forma de sistemas de apoio à tomada de decisão, para que seja efetivamente utilizado, auxiliando no alcance dos objetivos de negócio que motivaram a criação do projeto.

\end{enumerate}

\newpage

\subsection*{1 - Análise do negócio}

% diagrama

\begin{figure}[H]
	\includegraphics[scale=0.8]{img/CRISP-DM-Analise-de-negocio.png}
	\caption{Detalhamento da fase Análise de negócio}
	\label{img:CRISP-DM-analise-de-negocio}
\end{figure}

% Tarefas e saídas

\subsubsection*{\textbf{1.1 - Analisar os objetivos de negócio}}

O primeiro passo em um projeto de Mineração de Dados é analisar, sob uma perspectiva de negócio, o que realmente o cliente deseja alcançar. Normalmente o cliente possui vários objetivos concorrentes e restrições, que devem ser balanceados. O objetivo desta tarefa é descobrir fatores importantes do negócio que possam influenciar no resultado do projeto. Uma das consequências de se negligenciar esse passo é o projeto finalizar respondendo corretamente a perguntas erradas.

\subsubsection*{Background}

Registro das informações sobre o estado do negócio no início do projeto.

\subsubsection*{Objetivos do negócio}

Registra o objetivo primário do projeto, sob uma perspectiva de negócio, além de outras questões de negócio que o cliente objetiva esclarecer.

\subsubsection*{Critério de sucesso, sob uma perspectiva de negócio}

Registra os critérios para que o projeto seja considerado um sucesso, sob uma perspectiva de negócio.

\subsubsection*{\textbf{1.2 - Avaliar a situação}}

O objetivo desta tarefa é analisar mais detalhadamente de informações importantes para determinar os objetivos da Mineração de Dados e desenvolver um plano para o projeto. São analisadas informações como quais os recursos disponíveis, quais as restrições e quais as assunções.

\subsubsection*{Inventário de recursos}

Registra os recursos disponíveis para o projeto, como recursos humanos(especialistas do negócio, analistas de dados, técnicos de suporte), dados(arquivos, bases de dados operacionais, data warehouses), hardwares e softwares.

\subsubsection*{Requisitos, assunções e restrições}

Registra os requisitos do projeto, incluindo prazos, níveis de qualidade, segurança e aspectos legais.
Registra as assunções do projeto, sejam assunções que poderão ser verificadas a partir dos dados utilizados pelo projeto, sejam assunções que não poderão ser verificadas. No caso de assunções que não poderão ser verificadas, é importante que sejam registradas, principalmente caso possam afetar a validade dos resultados do projeto.
Registra as restrições do projeto, sejam restrições na disponibilidade de recursos, sejam restrições tecnológicas.

\subsubsection*{Riscos e planos de contingência}

Registra os eventos que, caso ocorram, poderão afetar os prazos ou a qualidade do projeto, bem como os planos de contingência, detalhando que ações devem ser executadas caso esses eventos ocorram.

\subsubsection*{Glossário}

Registra o conjunto de termos e seus significados que são relevantes para o projeto. Inclui tanto termos pertencentes à terminologia do negócio, quanto termos pertencentes à terminologia de Mineração de Dados.

\subsubsection*{Custos e benefícios}

Registra uma análise dos custos do projeto comparados com os potenciais benefícios para o negócio, caso o projeto alcance sucesso. Essa comparação deve ser o mais específico possível. Por exemplo, pode-se utilizar valores monetários em contextos comerciais.

\subsubsection*{\textbf{1.3 - Analisar os objetivos de Mineração de Dados}}

O objetivo desta tarefa é traduzir para objetivos, numa perspectiva de Mineração de Dados, os objetivos de negócio analisados na tarefa 1.1.

\subsubsection*{Objetivos da Mineração de Dados}

Registra os resultados objetivados pelo projeto para auxiliar no alcance dos objetivos de negócio.

\subsubsection*{Critério de sucesso, sob uma perspectiva de Mineração de Dados}

Registra, em termos técnicos, os critérios para determinar se o projeto alcançou sucesso.

\subsubsection*{\textbf{1.4 - Desenvolver um plano do projeto}}

Registra um plano para alcançar os objetivos de Mineração de Dados e então os objetivos de negócio. Inclui tanto as atividades a serem desenvolvidos, quanto as ferramentas e técnicas a serem utilizadas.

\subsubsection*{Plano do projeto}

Registra as atividades a serem desenvolvidas, incluindo duração, recursos necessários, entradas, saídas e dependências. É importante que sejam registradas as dependências e riscos das atividades e como podem impactar nos prazos.
Dado o aspecto iterativo de um projeto de Mineração de Dados, o plano de projeto é um documento dinâmico, sendo recomendado que ao fim de cada fase ele seja revisado e atualizado.

\subsubsection*{Avaliação inicial de ferramentas e técnicas}

Registra a avaliação de ferramentas e técnicas para a escolha de quais serão utilizadas no projeto. É importante que essa análise seja realizada no início do projeto, dado que a escolha das ferramentas e técnicas podem influenciar todo o resto do projeto.

\newpage 

\subsection*{2 - Análise dos dados}

% diagrama 

\begin{figure}[H]
	\includegraphics[scale=0.8]{img/CRISP-DM-Analise-dos-dados.png}
	\caption{Detalhamento da fase Análise dos dados}
	\label{img:CRISP-DM-Analise-dos-dados}
\end{figure}

% tarefas e saídas

\subsubsection*{\textbf{2.1 - Coletar dados}}

O objetivo desta tarefa é realizar a coleta dos dados indicados nos recursos do projeto. Nesta tarefa estão inclusos tanto o trabalho de extração quanto de integração dos dados, caso provenham de fontes de dados diferentes.

\subsubsection*{Relatório da coleta inicial de dados}

Registra os conjuntos de dados coletados, suas localizações, os métodos utilizados na coleta e problemas, com respectivas soluções adotadas, que nela tenham ocorrido.

\subsubsection*{\textbf{2.2 - Descrever os dados}}

O objetivo desta tarefa é realizar uma análise estrutural dos dados, avaliando se eles satisfazem os requisitos do projeto.

\subsubsection*{Relatório descritivo dos dados}

Registra informações sobre os dados coletados, como formato, quantidade de registros e nomes de atributos.

\subsubsection*{\textbf{2.3 - Explorar os dados}}

O objetivo desta tarefa é realizar uma análise da distribuição dos dados, através de consultas, visualizações e técnicas de report\todo{report techniques...}. Nela estão inclusas análise da distribuição de atributos dos dados, análise do relacionamento entre pares de atributos, análise de subpopulações e análise estatística. 

\subsubsection*{Relatório de exploração dos dados}

Registra as descobertas dessa tarefa e impacto que causarão no projeto.

\subsubsection*{\textbf{2.4 - Analisar a qualidade dos dados}}

O objetivo desta tarefa é analisar a qualidade dos dados, verificando, por exemplo, se são completos(há registros para todos os casos necessários), se são corretos(ausência de erros), se há dados ausentes; e analisar soluções para os problemas de qualidade encontrados.

\subsubsection*{Relatório de qualidade dos dados}

Registra os resultados da análise de qualidade dos dados, indicando, os problemas de qualidade e as possíveis soluções.

\newpage

\subsection*{3 - Preparação dos dados}

% diagrama

\begin{figure}[H]
	\includegraphics[scale=0.8]{img/CRISP-DM-Preparacao-dos-dados.png}
	\caption{Detalhamento da fase Preparação dos dados}
	\label{img:CRISP-DM-Preparacao-dos-dados}
\end{figure}

% tarefas e saídas

\subsubsection*{Dataset}

Datasets produzidos por esta fase, a serem utilizados no desenvolvimento de modelos ou em análises.

\subsubsection*{Descrição do dataset}

Registra informações sobre o dataset produzido por esta fase.

\subsubsection*{\textbf{3.1 - Selecionar dados}}

O objetivo desta tarefa é selecionar um subconjunto de dados a serem utilizados nas análises posteriores. Essa seleção envolve tanto a seleção de registros quanto a seleção de atributos. A lista de critérios para essa seleção inclui relevância dos dados para os objetivos de Mineração de Dados, qualidade e restrições técnicas, como limite no volume dos dados ou nos tipos dos dados.

\subsubsection*{Motivos para inclusão/exclusão de dados}

Registra os motivos para inclusão e exclusão de dados.

\subsubsection*{3.2 - Limpar os dados}

O objetivo desta tarefa é produzir um dataset com nível de qualidade adequado para a aplicação das técnicas e modelos selecionados pelo projeto, resolvendo os problemas de qualidade analisados na tarefa Analisar a qualidade dos dados. Para tanto, atividades como seleção de subconjunto dos dados, inserção de valores default e estimação de valores ausentes poderão ser necessárias.

\subsubsection*{Relatório da limpeza dos dados}

Registra as alterações realizadas nos dados para resolver problemas de qualidade, indicando os motivos e possíveis consequências.

\subsubsection*{\textbf{3.3 - Construir dados}}

O objetivo desta tarefa é a criação de novos dados, sejam derivados de dados já existentes, sejam novos registros, criados, por exemplo, através de interpolação de outros registros já existentes.

\subsubsection*{Atributos derivados}

Registra os atributos que foram construídos a partir de outros já existentes. Por exemplo, área = altura * largura.

\subsubsection*{Registros gerados}

Registra a geração de novos registros. 

\subsubsection*{\textbf{3.4 - Integrar os dados}}

O objetivo desta tarefa é criar novos dados através da integração de dados de fontes diversas.

\subsubsection*{Dados integrados}

Esta saída é composta tanto pelos dados que foram gerados a partir da integração de dados de fontes diversas, quanto dados agregados.

\subsubsection*{\textbf{3.5 - Formatar os dados}}

O objetivo desta tarefa é realizar transformações nos dados que não alterem seus significados, necessárias para que os dados possam ser utilizados pelas ferramentas. Exemplos de transformações são mudança do formato do arquivo onde estão os dados, alteração na ordem das colunas ou alteração na ordem dos registros.

\subsubsection*{Dados formatados}

Registra as transformações realizadas nos dados, indicando motivos e possíveis consequências.

\newpage 

\subsection*{4 - Modelagem}

% diagrama

\begin{figure}[H]
	\includegraphics[scale=0.8]{img/CRISP-DM-Modelagem.png}
	\caption{Detalhamento da fase Modelagem}
	\label{img:CRISP-DM-Modelagem}
\end{figure}

% tarefas e saídas

\subsubsection*{\textbf{4.1 - Selecionar uma técnica de modelagem}}

O objetivo desta tarefa é selecionar uma técnica de modelagem a ser aplicada a um dataset gerado. No caso de várias técnicas de modelagem terem sido escolhidas para serem aplicadas, esta tarefa deve ser executada para cada uma delas.

\subsubsection*{Técnica de modelagem selecionada}

Registra informações sobre a técnica de modelagem selecionada.

\subsubsection*{Assunções da técnica de modelagem selecionada}

Registra as assunções feitas pela técnica de modelagem selecionada. Por exemplo, que todos os registros são independentes ou que todos os atributos possuem distribuição uniforme.

\subsubsection*{\textbf{4.2 - Desenvolver esquema de teste}}

O objetivo desta tarefa é desenvolver um procedimento ou mecanismo para testar a qualidade e validade do modelo a ser desenvolvido. Para tanto, deve-se decidir, por exemplo, sobre como os dados serão particionados em subconjuntos de treinamento e de teste e quais métricas serão utilizadas para avaliar o desempenho.

\subsubsection*{Esquema de teste}

Registra um plano para treinamento, teste e avaliação do modelo a ser desenvolvido.

\subsubsection*{\textbf{4.3 - Desenvolver modelos}}

O objetivo desta tarefa é aplicar a técnica de modelagem escolhida ao dataset desenvolvido na fase anterior.

\subsubsection*{Parâmetros dos modelos}

Registra os parâmetros utilizados pelos modelos desenvolvidos, bem como os motivos para suas escolhas.

\subsubsection*{Modelos}

Modelos desenvolvidos.

\subsubsection*{Descrição dos modelos}

Registra informações sobre os modelos desenvolvidos, como, por exemplo, como interpretá-los.

\subsubsection*{\textbf{4.4 - Avaliar os modelos}}

O objetivo desta tarefa é avaliar os modelos desenvolvidos sob uma perspectiva de Mineração de Dados, verificando se os critérios de sucesso de Mineração de Dados foram satisfeitos, se os resultados do testes foram satisfatórios. Os modelos desenvolvidos devem ser então comparados e ordenados de acordo com critérios de avaliação.

\subsubsection*{Avaliação dos modelos}

Registra os resultados da tarefa Avaliar o modelo, como a performance dos modelos desenvolvidos e uma ordem dos modelos de acordo com critérios de qualidade.

\subsubsection*{Parâmetros revisados dos modelos}

Registra alterações propostas em parâmetros dos modelos desenvolvidos de acordo com a avaliação dos modelos. Os parâmetros revisados servem para serem utilizados no desenvolvimento, em uma nova iteração, de novos modelos.

\newpage 

\subsection*{Avaliação}

\newpage 

\subsection*{Instalação}

\section{Ferramentas utilizadas}
\input{./capitulos/resultados.tex}
\input{./capitulos/conclusao.tex}

% ---
% Finaliza a parte no bookmark do PDF, para que se inicie o bookmark na raiz
% ---
\bookmarksetup{startatroot}% 
% ---

% ----------------------------------------------------------
% ELEMENTOS PÓS-TEXTUAIS
% ----------------------------------------------------------
\postextual

% ----------------------------------------------------------
% Referências bibliográficas
% ----------------------------------------------------------

\bibliography{referencias}
% ----------------------------------------------------------
% DEMAIS ELEMENTOS PÓS-TEXTUAIS
% ----------------------------------------------------------
% \include{postextuais}

\end{document}