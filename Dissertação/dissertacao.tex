\documentclass[a4paper]{article}

\usepackage[brazilian]{babel}
\usepackage[utf8]{inputenc}
\usepackage[T1]{fontenc}
\usepackage{lipsum}
\usepackage{todonotes}
\usepackage{natbib}

\title{Graduação como processo de negócio: contribuições de data mining}

\author{Abelardo Vieira Mota}

\date{\today}

\begin{document}
\maketitle

\begin{abstract}
\end{abstract}

\tableofcontents

\section{Introdução}

\subsection{Contextualização}
\begin{itemize}
\item administração gerencial
\item graduação como processo de negócio
\item gerenciamento
\item oportunidade pela revisão de currículo da graduação da computação
\end{itemize}

\subsection{Objetivos}
\begin{itemize}
\item aplicando ferramentas de mineração de processos:
\begin{itemize}
\item modelar a graduação na UFC como um processo de negócio
\item descobrir informações relevantes para a melhoria da graduação na UFC
\end{itemize}
\item desenvolver um processo de modelagem de processos de negócio auxiliado por mineração de processos
\end{itemize}

\subsection{Proposta de trabalho}
\begin{itemize}
\item modelar a graduação como processo de negócio
\item aplicar BPM
\item suportar com process mining
\end{itemize}

\subsection{Desafios}
\begin{itemize}
\item estrutura de dados
\item resistência do orientador a trabalhar com administração rs
\item dados sensíveis
\item baixa qualidade dos dados
\end{itemize}
\subsection{Organização do trabalho}
\todo{apenas no fim}
\section{Graduação}

\subsection{Definições}
\todo{what}

\subsection{Fontes de informações}
\begin{itemize}
\item documentos normativos
\item pessoas
\item bases de dados
\end{itemize}

\section{Processo de negócio}
\subsection{Definição}
Dentro de uma organização, entendida como \todo{definição}, um conjunto de atividades são executadas a fim de que alguns resultados sejam alcançados.\todo{exemplo}
 
De acordo com \cite{BPM_CBOK}, um processo de negócio é ``uma agregação de atividades e comportamentos executados por humanos ou máquinas para alcançar um ou mais resultados'', sendo negócio definido como ``pessoas que interagem para executar um conjunto de atividades de entrega de valor para os clientes e gerar retorno às partes interessadas''. 

Da definição de processo de negócio, os seguintes elementos podem ser extraídos:
\begin{itemize}
\item atividades
\item comportamentos
\item humanos
\item máquinas
\item resultados
\end{itemize}

\todo{processo de negócio é uma visão sobre como o trabalho ocorre em um negócio}
\todo{benefícios da visão de processos de negócio}
\begin{itemize}
\item corpo de conhecimento já desenvolvido(BPM, BPMN, Process Mining)
\item organização entender como funciona
\item ponto de partida para melhorias/ajustes a alterações impostas
\item identificação de causas de problemas
\item facilidade de especificação para soluções de software de suporte ao trabalho
\end{itemize}

\subsection{Formalização}
\cite{PM_form}
\subsection{Linguagens de modelagem}

\subsubsection{Petri Net}
\cite{PM_book}

\subsubsection{BPMN}
\begin{itemize}
\item sobre BPMN
\item BPMN pallete level 1
\item metodologia de modelagem do BPMN Method and Style
\end{itemize}

\section{Process Mining}

\cite{PM_book}

\subsection{Histórico}

\subsection{Formalização}

\subsubsection{Process Discovery}

\subsubsection{Process Conformance}

\subsubsection{Process Enhancement}

\subsection{Ferramentas}

\subsubsection{ProM}

\cite{ProM}

\subsubsection{Disco}

\cite{Disco}

\section{Graduação como processo de negócio}

\subsection{Modelagem}
\subsubsection{Informações de documentos normativos}
\subsubsection{Informações de especialistas/atores do processo}
\subsubsection{Informações extraídas das bases de dados}

\section{Análise do curso Computação da UFC}

\subsection{Contexto}

\subsection{Hipóteses}

\subsection{Análise da retenção de discentes}


\section{Conclusão e trabalhos futuros}


\bibliographystyle{plain}
\bibliography{dissertacao}

\end{document}