\documentclass[a4paper]{article}

\usepackage[english]{babel}
\usepackage[utf8]{inputenc}

\title{Currículo como processo de negócio: como process mining pode auxiliar a revisão de currículos de graduação}

\author{Abelardo Vieira Mota}

\date{\today}

\begin{document}
\maketitle

\begin{abstract}
Your abstract.
\end{abstract}

\section{Introdução}

\subsection{Contextualização}
	
\subsubsection{Revisão de currículos}
\subsubsection{Processo de negócio}
\subsubsection{Process Mining}

\subsection{Proposta de trabalho}

\section{Currículos de graduação}

\subsection{Definições}

\subsection{Estruturas de dados}

\subsection{Modelagem}

\section{Processos de negócio}

\subsection{Definição}

\subsection{Histórico}

\subsection{Formalização}

\subsection{Modelagem}
	
\subsubsection{Petri Net}
\subsubsection{Causal Net}
\subsubsection{BPMN}

\section{Process Mining}

\subsection{Histórico}

\subsection{Formalização}
\subsubsection{Process Discovery}
\subsubsection{Process Conformance}
\subsubsection{Process Enhancement}

\subsection{Ferramentas}
\subsubsection{ProM}
\subsubsection{Disco}

\section{Currículo como processo de negócio}

\subsection{Proposta}

\subsection{Formalização}

\section{Aplicação de Process Mining}

\subsection{Process Discovery}
\subsubsection{Hipóteses}
\subsubsection{Análise}
\subsubsection{Resultados}

\subsection{Process Conformance}
\subsubsection{Hipóteses}
\subsubsection{Análise}
\subsubsection{Resultados}

\subsection{Process Enhancement}
\subsubsection{Hipóteses}
\subsubsection{Análise}
\subsubsection{Resultados}

\section{Conclusão e trabalhos futuros}

\end{document}