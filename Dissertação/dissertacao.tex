\documentclass[a4paper]{article}

\usepackage[brazilian]{babel}
\usepackage[utf8]{inputenc}
\usepackage[T1]{fontenc}
\usepackage{lipsum}
\usepackage{todonotes}
\usepackage{natbib}
\usepackage{booktabs}

\title{Aplicação de mineração de dados ao problema de evasão de estudantes da UFC}

\author{Abelardo Vieira Mota}

\date{\today}

\begin{document}
\maketitle

\begin{abstract}
resumo
\end{abstract}

\tableofcontents

\section{Introdução}

\subsection{Contextualização}
O Programa de Apoio a Planos de Reestruturação e Expansão das Universidades Federais(REUNI) foi instituído pelo DECRETO Nº 6.096, DE 24 DE ABRIL DE 2007\todo{como cita isso?} e possui como uma de suas diretrizes:
\begin{quote}
I - redução das taxas de evasão, ocupação de vagas ociosas e aumento de vagas de ingresso, especialmente no período noturno;
\end{quote}
\todo{DECRETO Nº 6.096, DE 24 DE ABRIL DE 2007}

O problema de evasão de estudantes refere-se ao abandono do curso pelo estudante antes de sua conclusão. De acordo com o Anuário estatístico da UFC de 2014, ano base 2013, o indicador "Taxa de Sucesso na Graduação"(TS) \todo{figura}, definido como a proporção entre número de diplomados e número de ingressantes da graduação, esteve em 2013 com o menor valor desde que passou a ser monitorado. Já o indicador "Taxa de sucesso da graduação por curso" \todo{figura} possui valor médio \todo{calcular} e mínimo \todo{analisar} para o curso \todo{analisar}. Esses indicadores apontam para que o problema da evasão de estudantes é uma realidade na UFC. 

Os dados gerenciados pelos sistemas de informação da UFC podem conter informações que auxiliem o entendimento das causas desse problema e permitam que melhor sejam planejadas ações para solucioná-lo ou que estudantes com maior risco de evasão sejam identificados e recebam apoio da instituição. É nesse sentido, que a área de pesquisa denominada mineração de dados(data mining, em inglês), que estuda como extrair informações sobre dados, pode contribuir, fornecendo meios para a descoberta de informações relevantes a partir dos dados registrados pela UFC.


\todo{falar sobre os estudos sobre drop-out+data mining}

\subsection{Objetivos}
O presente trabalho objetiva estudar o problema de evasão de estudantes na UFC a partir dos dados que seus sistemas de informação gerenciam, avaliando a aplicabilidade de técnicas de mineração de dados. 

Para tanto é necessário que seja feito uma análise sobre a estrutura e qualidade dos dados disponíveis.

Também serão levantadas hipóteses sobre causas para o problema e será analisado se os dados as corroboram ou não.
hipótese sobre a estrutura do curriculo -> métricas com relação às turmas! por exemplo, distância de horário entre disciplinas do mesmo semestre -> no caso do aluno, para cada semestre calcular

\subsection{Organização do trabalho}

\section{Mineração de dados}
\subsection{Identificação de padrões}
\todo{sobre data mining - qual o super artigo básico?}
\todo{clustering, visualização}
\subsection{Predição}
\cite{Tom_mitchell} \cite{ML_debt} \cite{ML_know}
\todo{machine learning}
\bibliographystyle{plain}

\section{Evasão de estudantes}
\cite{EDM_pred_dropout} \cite{EDM_review_and_soa} \cite{EDM_retention} \cite{EDM_education}

\bibliography{dissertacao}

\end{document}