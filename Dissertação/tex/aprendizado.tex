Aprendizado de máquina é uma subárea de inteligência artificial que agrupa conhecimentos sobre algoritmos e técnicas que permitam que um programa melhore sua performance a partir de dados. Mais formalmente, \cite{Tom_mitchell} define que um programa aprende a partir de uma experiência E, com relação a uma classe de tarefas T e a uma medida de performance P, se sua performance em tarefas da classe T, medida por P, melhora com a experiência E.

Seja, por exemplo, o problema de autoria de textos, que consiste na identificação automatizada do autor de um texto. Uma das soluções desenvolvidas é a verificação da similaridade entre o texto em análise e um conjunto de textos cujos autores já sejam conhecidos, denominado conjunto de treino, sendo reportado como o autor aquele cujos textos contidos no conjunto de treino sejam mais similares ao texto em análise. Nesse exemplo, o elemento experiência é um conjunto de textos rotulados com seus respectivos autores; o elemento tarefa é a identificação do autor de um dado texto; o elemento medida de performance a proporção de textos cujos autores são corretamente identificados.


\todo{problemas com tarefa, medida de performance, experiencia menos comuns}

\todo{bla bla bla de machine learning}

\todo{definição informal de aprendizado de máquina}
\todo{definição formal de aprendizado de máquina}
\todo{métricas}
\todo{algoritmos}
\todo{métodos de teste}
\todo{aplicação}
\cite{ML_debt} \cite{ML_know}
\todo{machine learning}
\bibliographystyle{plain}
