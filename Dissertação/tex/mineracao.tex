\todo{sobre data mining - qual o super artigo básico?}
\subsection{Identificação de padrões}
\todo{clustering, visualização}
\subsection{Aprendizado de máquina}
Aprendizado de máquina é uma subárea de inteligência artificial que agrupa conhecimentos sobre algoritmos e técnicas que permitam que um programa melhore sua performance a partir de dados. Mais formalmente, \cite{Tom_mitchell} define o aprendizado de um programa por\todo{como por a definition?}

Definition: A computer program is said to learn from experience E with respect to some class of tasks T and performance measure P, if its performance at tasks in T, as measured by P, improves with experience E. 

Para a aplicação dos conhecimentos de aprendizado de máquina para solucionar um problema, esses três elementos, tarefa(task), medida de performance(performance measure) e experiência(experience) devem ser identificados.
\todo{por que não aplicar outras técnicas?}
\todo{exemplos}
Reconhecimento de dígito manualmente escrito

O problema de reconhecimento de dígito manualmente escrito consiste na identificação automatizada do valor de um dígito manualmente escrito contido em uma imagem.

Tarefa: dada uma imagem contendo um dígito manualmente escrito, identificar qual o valor do dígito nela contido.
Experiência: arquivo de imagem contendo um dígito manualmente escrito e o valor do dígito nele contido.
Medida de performance: a proporção de dígitos corretamente identificados. 

Identificação de autoria de textos

O problema de identificação de autoria de textos consiste na identificação automatizada do autor de um texto.

Tarefa: dado um texto, identificar qual seu autor.
Experiência: texto e respectivo autor.
Medida de performance: a proporção de autores corretamente identificados.

\todo{problemas com tarefa, medida de performance, experiencia menos comuns}

\todo{bla bla bla de machine learning}

\todo{definição informal de aprendizado de máquina}
\todo{definição formal de aprendizado de máquina}
\todo{métricas}
\todo{algoritmos}
\todo{métodos de teste}
\todo{aplicação}
\cite{ML_debt} \cite{ML_know}
\todo{machine learning}
\bibliographystyle{plain}
