\todo{tabela com dados dos estudos analisados - ver fichamento.xlsx}
\todo{descrever, em termos gerais, cada estudo}
Em \cite{EDM_pred_dropout} são aplicados algoritmos de aprendizado de máquina a dados de estudantes do Eletrical Engineering department, Eindhoven University of Technology , com o objetivo de identificar estudantes em grupos de risco de evasão. É relatado que esse departamento já avaliava os estudantes com relação ao risco de evasão, mas de forma subjetiva. O estudo ressalta o maior custo da ocorrência de falsos negativos que de falsos positivos na identificação de estudantes com risco de evasão. Ocorre que, argumenta-se, há prejuízo maior em não oferecer apoio a um estudante com risco de evasão do que oferecer, desnecessariamente, apoio a um estudante sem tal risco. O estudo faz uso então de uma matriz de custo, com o classificador CostSensitiveClassifier, do Weka, obtendo melhores resultados, com relação a ocorrência de falsos negativos, mas com perdas de acurácia. 

\cite{EDM_review_and_soa} 
\cite{EDM_retention} 
\cite{EDM_education}
\cite{EDM_dropout_rates}
\cite{EDM_ufrj}
\cite{EDM_ufrj2}
\cite{EDM_brasil}