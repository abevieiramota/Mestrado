\begin{abstract}
% contexto - resultado - conclusão

A evasão de discentes de cursos de ensino superior é um problema que tem recebido atenção do governo federal, dadas as altas taxas observadas nos últimos anos e por implicar em desperdício de recursos e de tempo. Esse problema tem sido objeto de estudos que buscam, a partir de dados, construir modelos de predição que possam ser utilizados para identificar discentes com grande probabilidade de abandonarem seus cursos, de forma que ações possam ser executadas a fim de diminuir essa probabilidade. Uma das ferramentas utilizadas para construção desses modelos é Aprendizado de Máquina, subárea de Inteligência Artificial composta por algoritmos e técnicas que permitem que um programa melhore sua performance a partir de dados. Este trabalho objetiva avaliar a aplicabilidade de Aprendizado de Máquina ao problema de evasão de discentes da UFC.

 \vspace{\onelineskip}
    
 \noindent
 \textbf{Palavras-chaves}: Aprendizado de máquina. Evasão. Mineração de dados.
 
\end{abstract}