\chapter{Evasão de discentes}

% avaliação sobre todos os estudos
% TODO

% Predicting Students Drop Out: A Case Study \cite{Predicting_Students}

Em \cite{Predicting_Students} são aplicados algoritmos de aprendizado de máquina a dados de discentes do Electrical Engineering department, Eindhoven University of Technology, considerando o período de 2000 a 2009, com o objetivo de identificar discentes em grupos de risco de evasão. É relatado que esse departamento já avaliava os discentes com relação ao risco de evasão, mas de forma subjetiva. O estudo ressalta o maior custo da ocorrência de falsos negativos que de falsos positivos na identificação de discentes com risco de evasão. Ocorre que, argumenta-se, há prejuízo maior em não oferecer apoio a um discente com risco de evasão do que oferecer, desnecessariamente, apoio a um discente sem tal risco. O estudo faz uso então de uma matriz de custo, com o classificador CostSensitiveClassifier, do Weka, obtendo melhores resultados, com relação a ocorrência de falsos negativos, mas com perdas de acurácia. 

% Evaluating performance and dropouts of undergraduates using educational data mining \cite{EDM_ufrj}

Em \cite{EDM_ufrj} são aplicados algoritmos de aprendizado de máquina a dados de discentes de seis cursos da Universidade Federal do Rio de Janeiro, com o objetivo de identificar discentes que não terão pelo menos uma aprovação no segundo semestre de seus cursos. Os cursos considerados foram: Direito, Farmácia, Física, Engenharia Civil, Engenharia Mecânica, Engenharia de Produção. É indicado que esses cursos foram escolhidos por pertencerem a departamentos distintos, com perfis de discentes distintos. É observado também que tais cursos diferem com relação à quantidade de discentes ingressantes, à taxa de evasão registrada e à efetividade de certas práticas de ensino. Para cada curso são desenvolvidas uma base de dados de treinamento e uma base de dados de teste, composta pelos dados de seus discentes de primeiro semestre. Para a base de dados de treinamento foram utilizados os dados dos anos pares(de 1994.1 a 2008.1), já para a de teste foram utilizados os dados dos anos ímpares(de 1995.1 a 2009.1). Foram utilizados os algoritmos Naive Bayes, Multilayer Perceptron, Support Vector Machine e Decision Tree.

% Educational Data Mining with Focus on Dropout Rates \cite{EDM_dropout_rates}
% TODO

% Identificação dos Fatores que Influenciam a Evasão em Cursos de Graduação \cite{EDM_ufrj2}
% TODO

% Analysis of Student Data for Retention Using Data Mining Techniques \cite{EDM_retention}
% TODO

% Tabelas por atributos
% Definição de evasão

Definições de evasão:

\begin{itemize}
\item após prazo T conseguiu performance P
\end{itemize}

\begin{table}
\begin{tabular}{p{0.5\linewidth}m{0.2\linewidth}p{0.3\linewidth}}
\toprule
                            Artigo &  Período(em semestres) & Medida de performance\\
\midrule
  \cite{Predicting_Students} &  6 & Se conseguiu a propedeusa \\
\bottomrule
\end{tabular}
\caption{Definição de evasão}
\label{table:study_attributes_definicao}
\end{table}

% Base de dados

Características da base de dados:

\begin{table}
\begin{tabular}{p{0.5\linewidth}p{0.2\linewidth}c}
\toprule
                            Artigo &  Quantidade de instâncias & Período\\
\midrule
  \cite{Predicting_Students} &  648 & 2000-2009\\
  &  495 & 2000-2009\\
  &  516 & 2000-2009\\
  &  516 & 2000-2009\\
\bottomrule
\end{tabular}
\caption{Tamanho do dataset}
\label{table:study_attributes_dataset}
\end{table}

% Features
% TODO taxonomia de features

% Algorithms
% TODO ver como faz o tabelão do mal

% Evaluation

% Ferramentas

% Resultados

% Conclusões
