\chapter{Evasão de discentes}

% avaliação sobre todos os estudos
% TODO

% Predicting Students Drop Out: A Case Study \cite{Predicting_Students_Drop_Out}

Em \cite{Predicting_Students} são aplicados algoritmos de aprendizado de máquina a dados de discentes do Electrical Engineering department, Eindhoven University of Technology, com o objetivo de identificar discentes em grupos de risco de evasão. É relatado que esse departamento já avaliava os discentes com relação ao risco de evasão, mas de forma subjetiva. O estudo ressalta o maior custo da ocorrência de falsos negativos que de falsos positivos na identificação de discentes com risco de evasão. Ocorre que, argumenta-se, há prejuízo maior em não oferecer apoio a um discente com risco de evasão do que oferecer, desnecessariamente, apoio a um discente sem tal risco. O estudo faz uso então de uma matriz de custo, com o classificador CostSensitiveClassifier, do Weka, obtendo melhores resultados, com relação a ocorrência de falsos negativos, mas com perdas de acurácia. 

% Evaluating performance and dropouts of undergraduates using educational data mining \cite{EDM_ufrj}
% TODO

% Educational Data Mining with Focus on Dropout Rates \cite{EDM_dropout_rates}
% TODO

% Identificação dos Fatores que Influenciam a Evasão em Cursos de Graduação \cite{EDM_ufrj2}
% TODO

% Analysis of Student Data for Retention Using Data Mining Techniques \cite{EDM_retention}
% TODO

% Tabelas por atributos
% TODO