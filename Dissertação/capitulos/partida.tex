\chapter{Pontos de partida}

Como pontos de partida, consideraremos:

\begin{itemize}

\item Analisar as múltiplas definições de evasão de discentes, suas peculiaridades, as causas atribuídas pela literatura e a aplicabilidade de técnicas de aprendizado de máquina.

\item Analisar o problema da predição de aluno com alta probabilidade de evasão do curso a partir de dados referentes a sua vida pré universidade e dados de seu desempenho no primeiro semestre do curso. Esse problema é relevante àqueles interessados na diminuição de taxas de evasão de cursos a partir de atividades voltada a discentes em grupo de risco de evasão.

\item Analisar o problema da predição do desempenho de um discente em uma disciplina a partir de dados sobre seu histórico como discente. Esse problema é relevante àqueles interessados na melhoria do desempenho de discentes em uma determinada disciplina a partir de atividades a serem iniciadas antes de o discente efetivamente cursá-la.

\item Analisar o problema da predição da taxa de evasão de um curso a partir de seus dados, como, por exemplo, sua estrutura curricular. Esse problema é relevante ao processo de desenho ou redesenho de um curso, sua solução podendo ser utilizado como guia para decisões acerca do curso.

\end{itemize}

