\chapter{Introdução}

% O que é o problema

% [Background]
O fenômeno evasão de discente consiste na interrupção de um processo de aprendizado de um discente antes de sua conclusão. Por exemplo, um discente que abandonou o curso de Computação, na UFC, no qual estava matriculado havia dois anos, pois precisou trabalhar para sustentar sua família e os horários das disciplinas eram incompatíveis com os horários do trabalho. Deste exemplo pode-se observar alguns dos atributos do fenômeno: o agente que interrompeu o processo, o discente, o curso, a instituição de ensino superior(IES), o tempo cursado e o motivo.

% # discutir mais sobre o fenômeno -> números no brasil, na ufc etc
% dados de qtd evasão na UFC e no Brasil

% Consequências
Sob a perspectiva de que o processo de aprendizado é um investimento e que o resultado esperado é a sua conclusão, o fenômeno evasão de discente pode ser considerado um problema: para a sociedade, com a frustração da expectativa de formação de profissionais e pesquisadores qualificados; para a instituição de ensino, caso tenha realizado investimentos em infraestrutura e em recursos humanos para atender a uma quantidade esperada de discentes ativos maior que a real, ocorrendo desperdício de recursos; caso seu orçamento seja ou função da quantidade de discentes ativos, no caso das instituições de ensino particulares, ou função da quantidade de discentes diplomados, no caso das instituições de ensino superior públicas\todo{validar o impacto da evasão no orçamento das IFES}; para o indivíduo que investiu tempo, dinheiro e dedicação, mas não terá os benefícios da conclusão da graduação, crítica no caso das profissões que exigem, para serem exercidas, diploma de graduação.

% motivação para resolver o problema

O estudo da evasão de discentes é motivado não apenas pelos problemas que dela podem decorrer mas também por diretrizes dos diversos níveis administrativos envolvidos com o processo.

No nível federal, a redução da ocorrência desse fenômeno faz parte de uma das diretrizes do Programa de Apoio a Planos de Reestruturação e Expansão das Universidades Federais(REUNI), instituído pelo Decreto nº 6.096, de 24 de abril de 2007\cite{reuni}:

\begin{citacao}
I - redução das taxas de evasão, ocupação de vagas ociosas e aumento de vagas de ingresso, especialmente no período noturno;
\end{citacao}

Na UFC, o instrumento de planejamento Plano de Desenvolvimento Institucional\cite{pdi_ufc}, para o período de 2013 a 2017, apresenta como um dos objetivos da política de assistência estudantil a redução da evasão; o programa de gestão da chapa eleita para reitoria no período de 2015 a 2019\cite{henry} apresenta um conjunto de propostas que possuem como um dos objetivos a redução dos índices de evasão; o planejamento estratégico do Centro de Tecnologia da UFC para o período de 2015 a 2025 propõe a criação de uma equipe de apoio pedagógico para atuar no combate a problemas relacionados à evasão de discentes.

% Soluções

Para alcançar o objetivo de reduzir a evasão de discentes, \cite{evasao_panorama2} propõe um conjunto de sete ações:

\begin{enumerate}

\item Estabelecer um grupo de trabalho encarregado de reduzir a evasão
\item Avaliar as estatísticas da evasão
\item Determinar as causas da evasão
\item Estimular a visão da IES centrada no discente
\item Criar condições que atendam aos objetivos que atraíram os discentes
\item Tornar o ambiente e o trânsito na IES agradáveis aos discentes
\item Criar programa de aconselhamento e orientação dos discentes

\end{enumerate}

Para o desenvolvimento de ações de aconselhamento e orientação dos discentes, faz-se necessário identificar aqueles discentes que delas poderão se beneficiar: discentes com tendência de abandonarem a graduação ou de serem jubilados. Uma das formas de identificá-los é através da observação e avaliação de outros atores do processo, como os professores e a coordenação do curso. Essa forma é limitada pelo conjunto de dados acessíveis ao observador, como é o caso, por exemplo, da quantidade limitada, em relação ao total, de discentes aos quais um professor tem acesso; e pela capacidade limitada de processamento de informações que um observador humano tem. A utilização de processamento computacional permite superar essas limitações ao permitir a análise de grandes quantidades de dados.

A área Aprendizado de Máquina estuda o desenvolvimento de algoritmos e técnicas que permitam que um programa melhore sua performance a partir do uso de dados. Aplicações dessa área englobam sistemas de identificação de spams, de reconhecimento de voz, de reconhecimento de caracteres, de diagnóstico médico, dentre outros. 

 

\todo[inline]{falar de aprendizado de máquina}


% Objetivo

O presente trabalho objetiva avaliar a aplicabilidade de técnicas de Aprendizado de Máquina ao problema de evasão de discentes da UFC.