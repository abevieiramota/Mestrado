\chapter{Introdução}

% O que é o problema

% [Background]
O fenômeno evasão de discente consiste na interrupção de um processo de aprendizado de um discente antes de sua conclusão. Por exemplo, um discente que abandonou o curso de Computação, na UFC, no qual estava matriculado havia dois anos, pois precisou trabalhar para sustentar sua família e os horários das disciplinas eram incompatíveis com os horários do trabalho. Deste exemplo pode-se observar alguns dos atributos do fenômeno: o agente que interrompeu o processo, o discente, o curso, a instituição de ensino superior(IES), o tempo cursado e o motivo.

% # discutir mais sobre o fenômeno -> números no brasil, na ufc etc
% dados de qtd evasão na UFC e no Brasil

% Consequências
Sob a perspectiva de que o processo de aprendizado é um investimento e que o resultado esperado é a sua conclusão, o fenômeno evasão de discente pode ser considerado um problema: para a sociedade, com a frustração da expectativa de formação de profissionais e pesquisadores qualificados; para a instituição de ensino, caso tenha realizado investimentos em infraestrutura e em recursos humanos para atender a uma quantidade esperada de discentes ativos maior que a real, ocorrendo desperdício de recursos; caso seu orçamento seja ou função da quantidade de discentes ativos, no caso das instituições de ensino particulares, ou função da quantidade de discentes diplomados, no caso das instituições de ensino superior públicas\todo{validar o impacto da evasão no orçamento das IFES}; para o indivíduo que investiu tempo, dinheiro e dedicação, mas não terá os benefícios da conclusão da graduação, crítica no caso das profissões que exigem, para serem exercidas, diploma de graduação.

Em \cite{evasao_global} foi estimado um prejuízo econômico, decorrente da evasão de discentes de graduação no período de 2007 a 2012, para a UFPB, de R\$ 415.032.704,52. A estimativa considera perdidos os recursos financeiros investidos para manutenção do discente que não concluiu a graduação, utilizando a fórmula:

$Perda\ Anual=n\_evadidos \times t\_permanencia \times v\_aluno$

onde:
\begin{itemize}
\item $n\_evadidos$ representa a quantidade, por ano, média de discentes que evadiram no período, considerada a média aritmética do total de discentes que evadiram no período pela quantidade de anos do período.
\item $t\_permanencia$ representa o tempo de permanência esperado de um discente antes de evadir.
\item $v\_aluno$ representa o custo corrente com hospital universitário por aluno corrente, indicador definido pelo TCU\cite{indicadores_TCU}.
\end{itemize}

Aplicando a fórmula para os dados da UFC, obtemos o resultado \todo[inline]{estimar o prejuízo econômico da evasão para a UFC}.

% motivação para resolver o problema

O estudo da evasão de discentes é motivado não apenas pelos problemas que dela podem decorrer mas também por diretrizes dos diversos níveis administrativos envolvidos com o processo.

No nível federal, redução da ocorrência desse fenômeno faz parte de uma das diretrizes do Programa de Apoio a Planos de Reestruturação e Expansão das Universidades Federais(REUNI), instituído pelo Decreto nº 6.096, de 24 de abril de 2007\cite{reuni}:
\begin{quote}
I - redução das taxas de evasão, ocupação de vagas ociosas e aumento de vagas de ingresso, especialmente no período noturno;
\end{quote}

Na UFC, o instrumento de planejamento Plano de Desenvolvimento Institucional\cite{pdi_ufc}, para o período de 2013 a 2017, apresenta como um dos objetivos da política de assistência estudantil a redução da evasão; o programa de gestão da chapa eleita para reitoria no período de 2015 a 2019\cite{henry} apresenta um conjunto de propostas que possuem como um dos objetivos a redução dos índices de evasão; o planejamento estratégico do Centro de Tecnologia da UFC para o período de 2015 a 2025 propõe a criação de uma equipe de apoio pedagógico para atuar no combate a problemas relacionados à evasão de discentes.


% Soluções

Em \cite{evasao_panorama2} são apresentadas sete ações que ajudam a diminuir a ocorrência de evasão de discentes:

\begin{enumerate}

\item Estabelecer um grupo de trabalho encarregado de reduzir a evasão
\item Avaliar as estatísticas da evasão
\item Determinar as causas da evasão
\item Estimular a visão da IES centrada no aluno
\item Criar condições que atendam aos objetivos que atraíram os alunos
\item Tornar o ambiente e o trânsito na IES agradáveis aos alunos
\item Criar programa de aconselhamento e orientação dos aluno

\end{enumerate}

Estas ações podem ser beneficiadas pela utilização de ferramentas de Aprendizado de Máquina, subárea de Inteligência Artificial, que estuda o desenvolvimento de programas cujas performances melhorem a partir de dados. 

% argumentar que é um problema complexo
% tese de Tinto de que normalmente é a combinação de pequenas causas

% Predição
Para diminuir as taxas de evasão, uma das estratégias adotadas é a identificação precoce de discentes com grande tendência para abandonarem seus cursos e a execução de ações que minimizem tal tendência. A identificação pode ser conduzida por observação do comportamento e resultados dos discentes, de forma subjetiva, pelos docentes e coordenadores de cursos, por exemplo. Em estudo realizado no departamento de engenharia elétrica da Eindhoven University of Technology\cite{Predicting_Students}, é relatado que em dezembro os discentes desse departamente recebem um aviso informando se são ou não aconselhados a continuarem no curso. Esse aviso é baseado na performance do discente no curso e em informações obtidas de professores do primeiro semestre e de discentes monitores. É relatado que o aviso parece ter bastante acurácia: geralmente discentes aconselhados a continuarem têm sucesso no próximo ano do curso, enquanto aqueles desaconselhados geralmente não continuam no curso.
Dois problemas decorrem dessa forma de identificação: sendo conduzida por pessoas, essa forma de identificação é limitada pelo conjunto de observações as quais o observador tem acesso; sendo subjetiva, seus resultados podem sofrer resistência para serem aceitos. A utilização de técnicas de aprendizado de máquina como forma de identificação pode contornar esses problemas, por, primeiro, fazer uso de dados registrados por sistemas de informação, provavelmente contendo informações mais amplas que as que uma pessoa pode observar; segundo, por fazer maior uso de dados registrados, sendo aceita mais facilmente como identificação objetiva. Nesse estudo foram utilizados diversos algoritmos de aprendizado de máquina com o objetivo de tentar detectar que um estudante irá abandonar seu curso. Foram utilizadas informações de discente referentes tanto ao período anterior ao seu ingresso na universidade, quanto ao posterior.


% Objetivo

O presente trabalho objetiva avaliar a aplicabilidade de técnicas de aprendizado de máquina ao problema de evasão de discentes na UFC.