\chapter{Aprendizado de máquina}

% [Definição]
\section{Definição}

Aprendizado de máquina é uma subárea de Inteligência Artificial que agrupa conhecimentos sobre algoritmos e técnicas que permitam que um programa melhore sua performance a partir de dados. Mais formalmente:

\begin{citacao}
Um programa aprende a partir de uma experiência E, com relação a uma classe de tarefas T e a uma medida de performance P, se sua performance em tarefas da classe T, medida por P, melhora com a experiência E. \cite[p.2, tradução nossa]{Tom_mitchell}
\end{citacao}

\todo{expandir}
% [Exemplo]
\section{Modelagem}

\cite{Tom_mitchell}, para exemplificar a modelagem de um programa com uma abordagem de aprendizado de máquina, apresenta uma sequência de passos para desenvolver um programa que aprenda a jogar xadrez, a ser utilizado para disputar um campeonato mundial de xadrez.

% Escolha da experiência
\subsection*{Escolha da experiência}

O primeiro passo é escolher a experiência a partir da qual o programa irá aprender, denominada experiência de treinamento. \cite{Tom_mitchell} classifica os tipos de experiência a partir de três atributos: feedback, se direto ou indireto com relação a como o programa será utilizado; nível de controle que há sobre a experiência; e representatividade. É ressaltado que o tipo de experiência utilizada pelo programa pode ter impacto significativo no sucesso ou falha em seu aprendizado. 

O atributo feedback representa quão direta é a informação fornecida pela experiência para o problema em questão. 
Com relação a esse atributo, o elemento experiência pode ser classificado como experiência de feedback direto e experiência de feedback indireto. Por exemplo, a tupla estado do tabuleiro e melhor movimento possível a partir desse estado é classificada como experiência de feedback direto: o programa irá atuar realizando movimentos e esse tipo de experiência informa diretamente qual o melhor movimento. Já a tupla sequência de movimentos de uma partida e seu resultado final é classificada como experiência de feedback indireto: o resultado final da partida não fornece informação direta sobre a qualidade dos movimentos que nela foram executados. A atividade de determinar o grau de influência que elementos de uma experiência de feedback indireto têm sobre o resultado é denominada credit assignment. \todo{traduzir}

O atributo nível de controle representa quanto de controle é possível ter sobre a captura da experiência. Com relação a esse atributo, o elemento experiência pode ser classificado como experiência selecionada por especialista, experiência sugerida pelo programa e analisada por um especialista e experiência selecionada e analisada pelo programa. Por exemplo, a experiência será do tipo selecionada por especialista se houve um jogador experiente de xadrez que selecionou estados de tabuleiro e indicou que melhores movimentos poderão ser feitos; será do tipo sugerida pelo programa e analisada por um especialista se o próprio programa selecionar estados de tabuleiro para serem analisadas por um jogador experiente de xadrez; será do tipo selecionada e analisada pelo programa se o programa utilizar o resultado de partidas que disputar consigo mesmo.

O atributo representatividade indica quão bem a experiência reflete a realidade. Com relação a esse atributo, o elemento experiência pode ser classificado em representativo, se sua distribuição representar a distribuição dos exemplos com os quais o programa efetivamente será utilizado, e não representativo, caso contrário. Por exemplo, a experiência não irá representar a realidade caso esteja limitada ao conjunto de partidas de apenas um jogador: considerando que o programa será utilizado em um campeonato mundial, do qual participam jogadores diversos, com estilos de jogo diversos, é capaz de o programa, treinado com essa experiência, depare com estados de tabuleiro que não encontrou no treinamento. \cite{Tom_mitchell} ressalta que muito da teoria de aprendizado de máquina depende da assunção de que a experiência utilizada no treinamento reflete a realidade.

% Escolha da função alvo/target function
\subsection*{Escolha da função alvo}

O próximo passo é a escolha do tipo de conhecimento que deverá ser aprendido, representado por uma função denominada função alvo, e como ele será utilizado pelo programa. Considerando que o programa irá atuar como um jogador de xadrez, uma possível função a ser considerada é uma cujo domínio seja o conjunto de estados de tabuleiro e que retorne o melhor movimento a partir do estado de tabuleiro informado. Esse tipo de conhecimento depende da assunção de que, dado um estado de tabuleiro, existe um melhor movimento a ser executado. O problema de aprendizado dessa função depende, portanto, do problema de determinar quão um movimento influencia no resultado final de uma partida. \todo{discutir}
\todo{qual a escolhida no Mitchell? Lembrar de apresentar o exemplo}

Outro tipo de conhecimento é uma função que tenha como domínio o conjunto de estados de tabuleiro e retorne um número real, indicando quão bom o estado de tabuleiro informado é. 

O programa irá jogar verificando qual estado de tabuleiro maximiza o valor da função, considerando o conjunto de estados de tabuleiro que podem ser alcançados a partir do estado atual do tabuleiro e de todas jogadas válidas.

% Escolha de uma representação para a função alvo
\subsection*{Escolha de uma representação para a função alvo}

Após a escolha do tipo de conhecimento que deverá ser aprendido, é necessário definir como esse conhecimento será representado. A função que associa um estado de tabuleiro a um número real pode assumir diversas formas: pode ser uma matriz contendo uma célula com um número real para cada estado de tabuleiro possível; pode ser um conjunto de regras que associe atributos do estado do tabuleiro a números reais; pode ser uma função polinomial de atributos do estado do tabuleiro em números reais etc. Para darmos continuidade ao detalhamento dos passos, escolhemos aqui uma representação de função simples: denominaremos por V a função que associa um estado de tabuleiro a um número real, calculada como combinação linear dos seguintes atributos do estado do tabuleiro:

\begin{itemize}
\item $x_1$: número de peças pretas no tabuleiro
\item $x_2$: número de peças brancas no tabuleiro
\item $x_3$: número de reis pretos no tabuleiro
\item $x_4$: número de reis brancos no tabuleiro
\item $x_5$: número de peças pretas ameaçadas por peças brancas no tabuleiro
\item $x_6$: número de peças brancas ameaçadas por peças pretas no tabuleiro
\end{itemize}

A função V pode ser representada então por:
\\
$V(t) = w_0 + w_1 x_1 + w_2 x_2 + w_3 x_3 + w_4 x_4 + w_5 x_5 + w_6 x_6$
\\
Os coeficientes $w_0$ a $w_6$ são parâmetros do programa, a serem ajustados no aprendizado.
\\
Resumindo os passos até aqui realizados, temos:

\begin{itemize}
\item Tarefa: jogar xadrez
\item Medida de performance: proporção de jogos do campeonato mundial de xadrez ganhos
\item Experiência de treinamento: partidas disputadas pelo programa contra ele mesmo
\item Função alvo: $V:EstadosDeTabuleiro \rightarrow \mathbb{R}$
\item Representação da função alvo: $V(t) = w_0 + w_1 x_1 + w_2 x_2 + w_3 x_3 + w_4 x_4 + w_5 x_5 + w_6 x_6$
\end{itemize}

% Escolha de um algoritmo de aproximação 
\subsection*{Escolha de um algoritmo de aproximação}

O próximo passo consiste na escolha de um algoritmo que, a partir de um conjunto de experiências, irá ajustar os parâmetros da representação da função a fim de aproximá-la da função alvo. Para tanto, é necessário um conjunto de dados de treinamento, composto por um par de estado de tabuleiro e o valor que a função deverá atribuir. Denotamos por $\langle b, V_{treino}(b) \rangle $ um dado para treinamento: $b$ é uma tupla contendo os atributos de um estado de tabuleiro e $V_{treino}(b)$ o valor a ele atribuído. Por exemplo:

$\langle \langle x_1 = 3, x_2 = 0, x_3 = 1, x_4 = 0, x_5 = 0, x_6 = 0 \rangle, +100\rangle$

Determinar os valores dos estados de tabuleiros utilizados no treinamento é tarefa fácil para os estados finais: pode-se definir dois valores, $max < min$, e atribuir aos estados finais de vitória o valor $max$ e aos de derrota o valor $min$. Já a determinação dos valores dos tabuleiros intermediários não é tão simples. O fato de uma partida ter sido ganha não implica necessariamente que todos os estados de tabuleiro nela percorridos devem receber um valor alto. Para definir tais valores pode-se utilizar uma regra de estimação. Por exemplo:

\begin{equation}
V_{treino}(b) \leftarrow V(sucessor(b))
\label{eq:critico}
\end{equation}

Essa regra atribui a um estado de tabuleiro de treinamento o valor que a função alvo estimada retorna para o estado de tabuleiro após a próxima jogada do oponente. Apesar de parecer estranho fazer uso de $V$, a função que se está estimando, para determinar os valores a serem utilizados para refiná-la, intuitivamente essa abordagem parece fazer sentido, por atribuir a um estado de tabuleiro um valor que é função de um estado que foi possível alcançar a partir dele. 

Após a determinação dos valores dos estados de tabuleiros contidos na experiência de treinamento, o conjunto de dados de treinamento pode ser utilizado. Para estimar a função alvo a partir dos dados de treinamento, dada a representação da função escolhida, é necessário agora determinarmos seus pesos, $w_0$ a $w_6$. Para tanto, primeiro definimos como mensurar quão bem a função estimada se adequa aos dados de treinamento. Uma medida comum é o erro quadrático, assim definido:

$E = \sum\limits_{\langle b, V_{treino}(b) \rangle \in \ dados\ de\ treinamento} {(V_{treino}(b) - V(b))}^2$

O problema de estimar a função alvo pode ser modelado então como o problema de encontrar os pesos $w_0$ a $w_6$ que minimizem o erro quadrático sobre os dados de treinamento. Um dos algoritmos que incrementalmente ajusta os pesos aos dados de treinamento, minimizando o erro quadrático é o Método dos Mínimos Quadrados. Esse algoritmo funciona ajustando, para cada dado de treinamento, os pesos da função na direção que minimiza o erro quadrático. Para tanto, atualiza iterativamente, para cada dado de treinamento, os pesos da função, incrementando com um valor proporcional a $(V_{treino}(b) - V(b))$: se o valor da função aplicado ao dado de treinamento, $V(b)$, é igual ao valor do dado de treinamento, $V_{treino}(b)$, o incremento será nulo; se o valor da função é maior que o do dado de treinamento, o incremento será negativo, fazendo com que, para o dado de treinamento em questão, o valor da função diminua; se o valor da função é menor que o do dado de treinamento, o incremento será positivo, fazendo com que, para o dado de treinamento em questão, o valor da função aumente.

% Conclusão
\subsection*{Conclusão}

Nessa sequência de passos podem ser identificados quatro elementos de um programa que aprende:

\begin{itemize}
\item Sistema de performance: elemento responsável pela utilização do conhecimento aprendido para resolver uma tarefa. No exemplo, será responsável por determinar qual a próxima jogada, dado um estado de tabuleiro, utilizando a função que foi aprendida. 
\item Crítico: elemento responsável por receber como entrada um dado de treinamento e informar a que valor deve ser associado. No exemplo, o crítico é representado pela equação \ref{eq:critico}.
\item Generalizador: elemento responsável por receber como entrada um conjunto de dados de treinamento e retornar uma função estimativa de uma função alvo. No exemplo, o generalizador é o Método dos Mínimos Quadrados.
\item Representação da função alvo: elemento que define a estrutura da função que será utilizada como estimativa da função alvo. No exemplo, foi utilizada como representação uma combinação linear.
\end{itemize}

Outras configurações para esses elementos foram desenvolvidas. Por exemplo, como representação da função alvo pode-se utilizar um grafo em estrutura de árvore, denominado árvore de decisão. Cada nó seu que não seja folha possui uma regra que associa um dado a um de seus nós filhos. Os nós folhas são associados a um valor. Seu funcionamento consiste em apresentar um dado à regra de um nó, inicialmente o nó raiz, e recursivamente aplicar esse procedimento ao nó filho ao qual a regra associa o dado, até que seja alcançado um nó folha, cujo resultado associado é então retornado como o valor da função.

\section{Modelos}
\todo{definir o que quer dizer por modelo}
representation + evaluation + optimization \cite{ML_know}

% [Aplicações]
\section{Aplicações}

\todo{melhorar}
De acordo com \cite{Mitchell_discipline}, o conhecimento sobre aprendizado de máquina pode ser aplicado a diversas áreas, como, por exemplo, a reconhecimento de voz; a visão computacional, sendo utilizado no desenvolvimento de sistemas de reconhecimento facial; a controle de robôs.