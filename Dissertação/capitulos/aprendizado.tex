\chapter{Aprendizado de máquina}

Aprendizado de máquina é uma subárea de Inteligência Artificial que agrupa conhecimentos sobre algoritmos e técnicas que permitam que um programa melhore sua performance a partir de dados. Mais formalmente, \cite[p.2, tradução nossa]{Tom_mitchell} define:

\begin{quote}
Um programa aprende a partir de uma experiência E, com relação a uma classe de tarefas T e a uma medida de performance P, se sua performance em tarefas da classe T, medida por P, melhora com a experiência E.
\end{quote}

Seja, por exemplo, o problema de identificação de autoria de textos. Uma das soluções desenvolvidas é a verificação da similaridade entre o texto em análise e um conjunto de textos cujos autores já sejam conhecidos, denominado conjunto de treino, sendo reportado como o autor aquele cujos textos contidos no conjunto de treino sejam mais similares ao texto em análise. Nesse exemplo, o elemento experiência é representado por um conjunto de textos rotulados com seus respectivos autores; o elemento tarefa é a identificação do autor de um texto; o elemento medida de performance é a proporção de textos cujos autores são corretamente identificados.



\cite{Mitchell_discipline}
\todo{bla bla bla de machine learning do Mitchell}

\todo{problemas com tarefa, medida de performance, experiencia menos comuns}

\todo{métricas}
\todo{algoritmos}
\todo{métodos de teste}
\todo{aplicação}
\cite{ML_debt} 
\cite{ML_know}
