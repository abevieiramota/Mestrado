\chapter{Metodologia}

\section{Processo CRISP-DM 1.0}
\todo[inline]{Por que um processo?}
% Reproducibilidade/padronização/etc
\cite{ML_know} \cite{balance-anarchy} \cite{ML_debt} \cite{replicability}

\todo[inline]{Por que CRISP-DM?}
\cite{CRISP-DM-KDD-SEMMA}

\todo[inline]{História do CRISP-DM}
CRISP-DM(CRoss-Industry Standard Process for Data Mining) é um processo de aplicação de Mineração de Dados, desenvolvido pelo CRISP-DM Special Interest Group e publicado em 2000. Foi concebido em 1996 por três empresas que utilizavam Mineração de Dados: DaimlerChrysler(à época Daimler-Benz), SPSS(à época ISL) e NCR; motivadas pela incerteza com relação à qualidade de seus trabalhos, pelo questionamento de se toda nova empresa que quiser aplicar Mineração de Dados terá que passar pelo aprendizado que passaram, baseado em tentativa e erro, e como garantir, para seus clientes, que Mineração de Dados era uma área suficientemente madura para ser incorporada a seus processos de negócio. Em 1999 foi publicado um draft do CRISP-DM versão 1.0, sendo aplicado pela DaimlerChrysler, SPSS e NCR a vários tipos de aplicações, indústrias e problemas de negócio, sendo considerado, então, validado suficientemente para ser publicado e distribuído\cite{CRISP-DM}.
\todo[inline]{E hoje em dia?}

\todo[inline]{Overview}

\subsection*{Business Understanding}
\subsection*{Data Understanding}
\subsection*{Data Preparation}
\subsection*{Modeling}
\subsection*{Evaluation}
\subsection*{Deployment}

\section{Ferramentas utilizadas}