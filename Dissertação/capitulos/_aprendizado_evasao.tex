\begin{comment}
% Escolha da medida de performance

O primeiro passo é a escolha da medida de performance. Um exemplo é a quantidade de discentes identificados que efetivamente abandonarem seus cursos. Uma questão surge: se o programa a ser desenvolvido for ser utilizado para identificar discentes que potencialmente abandonarão seus cursos, para os quais serão realizadas atividades para diminuição desse potencial, como contar precisamente quantos deles foram corretamente identificados? Como verificar que um discente que tinha a pretensão de abandonar o curso no momento da identificação pelo programa foi corretamente identificado se, após ações realizadas com a finalidade de diminuir seu potencial de evasão, ele concluiu o curso? A identificação pelo programa, seguido das ações, torna-se mais uma variável concorrente para o resultado do discente no curso. Outra questão relevante é o custo dos erros na identificação: considerando que identificar corretamente um discente com potencial de evasão implica em diminuição desse potencial, e que identificar incorretamente um discente sem potencial de evasão não implica em aumento desse potencial, concluímos que a performance do programa deve levar em consideração a diferença entre os custos de seus erros. 
% todo: evoluir sobre medidas de performance

% Escolha da experiência
O próximo passo é a escolha da experiência a partir da qual o programa irá aprender. \cite{Tom_mitchell} indica que o tipo de experiência utilizada pelo programa pode ter impacto significativo no sucesso ou falha em seu aprendizado. A literatura registra diversos tipos de experiência para o problema de evasão de estudantes: em \cite{Predicting_Students} são utilizadas informações sobre discentes e se, num prazo limite de três anos, foram aprovados em todas disciplinas do primeiro ano do curso. Já em \cite{EDM_ufrj} são utilizadas informações sobre discentes e se estes foram aprovados em pelo menos uma disciplina do segundo semestre de seus cursos. \cite{Tom_mitchell} classifica os tipos de experiência a partir de três atributos: se a experiência oferece feedback direto ou indireto com relação à medida de performance; o nível de controle que há sobre a experiência; e quão bem a experiência reflete a realidade.

%    feedback
O uso de informações sobre a aprovação de um discente em pelo menos uma disciplina do segundo semestre fornece feedback indireto para o problema de evasão de discente no curso: não se pode deduzir, a partir desse tipo de experiência, se o discente irá ou não abandonar seu curso. Já a informação de um discente sobre a conclusão de seu curso, considerando que tal situação não possa mais mudar, fornece feedback direto, mas apresenta uma desvantagem com relação ao primeiro tipo de experiência exemplificado: garantir que a situação de conclusão não irá mudar envolve a análise dos fatores que podem mudá-la. É necessário, por exemplo, garantir que o prazo limite para conclusão do curso já tenha expirado, o que pode impactar negativamente na quantidade de experiência disponível: para um curso, por exemplo, com prazo de conclusão máximo de 8 anos, só poderão ser considerados discentes cujo ingresso ocorreu há no mínimo 8 anos. Já o primeiro tipo de experiência permite a utilização de informações de discentes que ingressaram há no mínimo um ano, implicando em maior quantidade de experiência disponível.

\end{comment}