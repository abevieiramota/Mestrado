% ----------------------------------------------------------
% ELEMENTOS PRÉ-TEXTUAIS
% ----------------------------------------------------------
% \pretextual

% ---
% Capa
% ---
\imprimircapa

% ---
% Folha de rosto
% (o * indica que haverá a ficha bibliográfica)

\imprimirfolhaderosto

% ---
%\imprimirfolhaderosto*

% ---

% ---
% Inserir a ficha bibliografica
% ---

% Isto é um exemplo de Ficha Catalográfica, ou ``Dados internacionais de
% catalogação-na-publicação''. Você pode utilizar este modelo como referência. 
% Porém, provavelmente a biblioteca da sua universidade lhe fornecerá um PDF
% com a ficha catalográfica definitiva após a defesa do trabalho. Quando estiver
% com o documento, salve-o como PDF no diretório do seu projeto e substitua todo
% o conteúdo de implementação deste arquivo pelo comando abaixo:
%
% \begin{fichacatalografica}
%     \includepdf{fig_ficha_catalografica.pdf}
% \end{fichacatalografica}
%\begin{fichacatalografica}
%	\vspace*{\fill}					% Posição vertical
%	\hrule							% Linha horizontal
%	\begin{center}					% Minipage Centralizado
%	\begin{minipage}[c]{12.5cm}		% Largura
%	
%	\imprimirautor
%	
%	\hspace{0.5cm} \imprimirtitulo  / \imprimirautor. --
%	\imprimirlocal, \imprimirdata-
%	
%	\hspace{0.5cm} \pageref{LastPage} p. : il. (algumas color.) ; 30 cm.\\
%	
%	\hspace{0.5cm} \imprimirorientadorRotulo~\imprimirorientador\\
%	
%	\hspace{0.5cm}
%	\parbox[t]{\textwidth}{\imprimirtipotrabalho~--~\imprimirinstituicao,
%	\imprimirdata.}\\
%	
%	\hspace{0.5cm}
%		1. Palavra-chave1.
%		2. Palavra-chave2.
%		I. Orientador.
%		II. Universidade xxx.
%		III. Faculdade de xxx.
%		IV. Título\\ 			
%	
%	\hspace{8.75cm} CDU 02:141:005.7\\
%	
%	\end{minipage}
%	\end{center}
%	\hrule
%\end{fichacatalografica}

% ---


% ---
% Inserir folha de aprovação
% ---

% Isto é um exemplo de Folha de aprovação, elemento obrigatório da NBR
% 14724/2011 (seção 4.2.1.3). Você pode utilizar este modelo até a aprovação
% do trabalho. Após isso, substitua todo o conteúdo deste arquivo por uma
% imagem da página assinada pela banca com o comando abaixo:
%
% \includepdf{folhadeaprovacao_final.pdf}
%

%\begin{folhadeaprovacao}
%
%  \begin{center}
%    \vspace*{1cm}
%    {\ABNTEXchapterfont\large\imprimirautor}
%
%    \vspace*{\fill}\vspace*{\fill}
%    {\ABNTEXchapterfont\bfseries\Large\imprimirtitulo}
%    \vspace*{\fill}
%    
%    \hspace{.45\textwidth}
%    \begin{minipage}{.5\textwidth}
%        \imprimirpreambulo
%    \end{minipage}%
%    \vspace*{\fill}
%   \end{center}
    
%   Trabalho aprovado. \imprimirlocal, 24 de novembro de 2012:

%   \assinatura{\textbf{\imprimirorientador} \\ Orientador} 
%   \assinatura{\textbf{Professor} \\ Convidado 1}
%   \assinatura{\textbf{Professor} \\ Convidado 2}
   %\assinatura{\textbf{Professor} \\ Convidado 3}
   %\assinatura{\textbf{Professor} \\ Convidado 4}
      
%   \begin{center}
%    \vspace*{0.5cm}
%    {\large\imprimirlocal}
%    \par
%    {\large\imprimirdata}
%    \vspace*{1cm}
%  \end{center}
  
%\end{folhadeaprovacao}

% ---

% ---
% DEDICATÓRIA
% ---

% proposta sem dedicatória

%\begin{dedicatoria}
%   \vspace*{\fill}
%   \centering
%   \noindent
%   Dedico este trabalho.\vspace*{\fill}
%\end{dedicatoria}


% ---

% ---
% Agradecimentos
% ---
%\begin{agradecimentos}
%Agredicmentos..
%\end{agradecimentos}
% ---

% ---
% EPÍGRAFE
% ---

% proposta sem epígrafe

%\begin{epigrafe}
%    \vspace*{\fill}
%	\begin{flushright}
%		\textit{Essentially, all models are wrong, but some are useful.} \\
%		(George E. P. Box)
%	\end{flushright}
%\end{epigrafe}


% ---

% ---
% RESUMOS
% ---

% proposta sem resumos


% resumo em português
%\begin{abstract}

The ideal abstract will be brief, limited to one paragraph and no more than six ou seven sentences, to let readers scan it quickly for an overview of the paper's content.
 \vspace{\onelineskip}
    
 \noindent
 \textbf{Palavras-chaves}: Aprendizado de máquina. Evasão.
 
\end{abstract}

% resumo em inglês
%\begin{resumo}[Abstract]
%The ideal abstract will be brief, limited to one paragraph and no more than six ou seven sentences, to let readers scan it quickly for an overview of the paper's content.
% \vspace{\onelineskip}

% \noindent 
% \textbf{Key-words}: Machine Learning. Drop-out. Data Mining.
%\end{resumo}

% ---
% inserir lista de ilustrações
% ---
\pdfbookmark[0]{\listfigurename}{lof}
\listoffigures*
\cleardoublepage
% ---

% ---
% inserir lista de tabelas
% ---
\pdfbookmark[0]{\listtablename}{lot}
\listoftables*
\cleardoublepage
% ---

% ---
% inserir lista de abreviaturas e siglas
% A lista de Abreviaturas e Siglas pode ser facilmente montada com o pacote 
% nomencl. Abaixo segue um exemplo.
% ---
\nomenclature{Fig.}{Figura}
\nomenclature{$A_i$}{Area of the $i^{th}$ component} 
\nomenclature{456}{Isto é um número}
\nomenclature{123}{Isto é outro número}
\nomenclature{a}{primeira letra do alfabeto}
\nomenclature{lauro}{este é meu nome} 

\renewcommand{\nomname}{Lista de abreviaturas e siglas}
\pdfbookmark[0]{\nomname}{las}
\printnomenclature
\cleardoublepage
% ---

% ---
% inserir lista de símbolos
% ---
% O abnTeX2 não provê mecanismo para lista de símbolos.
% ---

% ---
% inserir o sumario
% ---
\pdfbookmark[0]{\contentsname}{toc}
\tableofcontents*
\cleardoublepage
% ---