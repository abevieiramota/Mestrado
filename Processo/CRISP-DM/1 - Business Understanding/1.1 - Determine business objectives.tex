\documentclass{article}
\usepackage{url}
\usepackage{todonotes}
\usepackage[utf8]{inputenc}	

\begin{document}

\section*{Background}

\subsection*{Site da UFC}

"A Universidade Federal do Ceará é uma autarquia vinculada ao Ministério da Educação. Nasceu como resultado de um amplo movimento de opinião pública. Foi criada pela Lei nº 2.373, em 16 de dezembro de 1954, e instalada em 25 de junho do ano seguinte.

No início, sob a direção de seu fundador, Prof. Antônio Martins Filho, era constituída pela Escola de Agronomia, Faculdade de Direito, Faculdade de Medicina e Faculdade de Farmácia e Odontologia.

Sediada em Fortaleza, Capital do Estado, a UFC é um braço do sistema do Ensino Superior do Ceará e sua atuação tem por base todo o território cearense, de forma a atender às diferentes escalas de exigências da sociedade.

A Universidade é composta de sete campi, denominados Campus do Benfica, Campus do Pici e Campus do Porangabuçu, todos localizados no município de Fortaleza (sede da UFC), além do Campus de Sobral, Campus de Quixadá, Campus de Crateús e Campus de Russas.

A Universidade Federal do Ceará, que há mais de 50 anos mantém o compromisso de servir à região, sem esquecer o caráter universal de sua produção, chega hoje com praticamente todas as áreas do conhecimento representadas em seus campi." \url{http://ufc.br/a-universidade}

\todo[inline]{tabela com cursos por campus + data de implantação}

\subsection*{Anuário estatístico 2014 base 2013}

\subsubsection*{Objetivos institucionais}

A UFC orienta sua atuação permanentemente no sentido de alcançar os seguintes objetivos:
\begin{enumerate}
\item
Promover a formação humana e profissional de seus estudantes, preparando-os para uma atuação responsável e construtiva na sociedade;
\item
Fomentar a geração de conhecimentos voltados para o desenvolvimento sustentável do Ceará e do Nordeste;
\item
Impulsionar o desenvolvimento, a produção e a preservação da cultura e das artes, com ênfase para as manifestações regionais.
\item
Promover a interação com a sociedade, através da difusão científica, tecnológica, artística e cultural e do desenvolvimento comunitário, sintonizados com as demandas sociais;
\item
Incentivar a capacitação permanente dos quadros docente e técnico-administrativo;
\item
Intensificar e ampliar as relações de parceria e intercâmbio com instituições nacionais e estrangeiras, governamentais e não governamentais;
\item
Buscar a profissionalização da gestão administrativa, apoiada em processos de planejamento e avaliação, executada com base em modelo organizacional flexível, eficiente e eficaz;
\item
Exercitar permanentemente o instituto da autonomia universitária superando restrições e estabelecendo novos parâmetros na gestão e nas relações institucionais;
\item
Assegurar a qualidade no desenvolvimento de todas as ações administrativas e acadêmicas;
\item
Distinguir-se como referência regional pela excelência acadêmica de suas ações nas áreas do ensino, geração do conhecimento e prestação de serviços à população, bem como na produção de arte e cultura.
\end{enumerate}
\textbf{p. 10}

\subsubsection*{Lema}

“O universal pelo regional” é o lema da UFC, instituição que busca centrar seu compromisso na solução dos problemas locais, sem esquecer o caráter universal de sua produção.
\textbf{p. 11}

\subsubsection*{Missão}

A missão da Universidade é formar profissionais da mais alta qualificação, gerar e difundir conhecimentos, preservar e divulgar os valores éticos, científicos, artísticos e culturais, constituindo-se em instituição estratégica para o desenvolvimento do Ceará, do Nordeste e do Brasil.
\textbf{p. 11}

\subsubsection*{Visão}

Consolidar-se como instituição de referência no ensino de graduação e pós-graduação (stricto e lato sensu), de preservação, geração e produção de ciência e tecnologia, e de integração com o meio, como forma de contribuir para a superação das desigualdades sociais e econômicas, por meio da promoção do desenvolvimento sustentável do Ceará, do Nordeste e do Brasil.
\textbf{p. 11}

\subsubsection*{Centros, faculdades e institutos}

\begin{enumerate}
\item 
Centro de Ciências
\item 
Centro de Humanidades
\item 
Centro de Tecnologia
\item 
Centro de Ciências Agrárias
\item 
Faculdade de Medicina
\item 
Faculdade de Farmácia, Odontologia e Enfermagem
\item 
Faculdade de Direito
\item 
Faculdade de Educação
\item 
Faculdade de Economia, Administração, Atuariais, Contabilidade e Secretariado Executivo
\item 
Instituto de Ciências do Mar (LABOMAR)
\item 
Instituto de Cultura e Arte (ICA)
\item 
Instituto de Educação Física e Esportes (IEFES)
\item 
Instituto Universidade Virtual – UFC Virtual
\end{enumerate}
\textbf{p. 13}

\subsubsection*{Quantitativos de bolsas por ano e modalidade}
\todo[inline]{tem na p. 38, mas é necessário solicitar em formato estruturado para utilizar. e-sic? alguém da prex, prpl? nívia?}
\textbf{p. 38}

\subsubsection*{Projetos e programas de inclusão social em 2013}
\todo[inline]{dados não estruturados -> esses projetos atuam na dimensão participação/inserção do discente}
\textbf{p. 43~44}

\subsubsection*{Unidade Universitária Federal de Educação Infantil - UUFEI}
\todo[inline]{suporte para discentes que possuam crianças - estranho serem apenas ~60, imaginava mais - p.47 diz serem 54\% filhos de discentes em 2013}
\textbf{p. 45~47}

\section*{Business objectives}

\section*{Business success criteria}

\cite{anuario_2014_base_2013}
\cite{pdi_ufc}

\bibliographystyle{acm}
\bibliography{../../../referencias}

\end{document}
