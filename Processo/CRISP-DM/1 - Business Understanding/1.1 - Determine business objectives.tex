\documentclass{report}
\usepackage{url}
\usepackage{todonotes}
\usepackage{booktabs}
\usepackage[utf8]{inputenc}	

\begin{document}

\listoftodos

\tableofcontents

\chapter{Background}

\section{Fonte: site da UFC}

"A Universidade Federal do Ceará é uma autarquia vinculada ao Ministério da Educação. Nasceu como resultado de um amplo movimento de opinião pública. Foi criada pela Lei nº 2.373, em 16 de dezembro de 1954, e instalada em 25 de junho do ano seguinte.

No início, sob a direção de seu fundador, Prof. Antônio Martins Filho, era constituída pela Escola de Agronomia, Faculdade de Direito, Faculdade de Medicina e Faculdade de Farmácia e Odontologia.

Sediada em Fortaleza, Capital do Estado, a UFC é um braço do sistema do Ensino Superior do Ceará e sua atuação tem por base todo o território cearense, de forma a atender às diferentes escalas de exigências da sociedade.

A Universidade é composta de sete campi, denominados Campus do Benfica, Campus do Pici e Campus do Porangabuçu, todos localizados no município de Fortaleza (sede da UFC), além do Campus de Sobral, Campus de Quixadá, Campus de Crateús e Campus de Russas.

A Universidade Federal do Ceará, que há mais de 50 anos mantém o compromisso de servir à região, sem esquecer o caráter universal de sua produção, chega hoje com praticamente todas as áreas do conhecimento representadas em seus campi." \url{http://ufc.br/a-universidade}

\section{Fonte: Anuário estatístico 2014 base 2013}

\cite{anuario_2014_base_2013}

\subsection{Objetivos institucionais}

A UFC orienta sua atuação permanentemente no sentido de alcançar os seguintes objetivos:
\begin{enumerate}
\item
Promover a formação humana e profissional de seus estudantes, preparando-os para uma atuação responsável e construtiva na sociedade;
\item
Fomentar a geração de conhecimentos voltados para o desenvolvimento sustentável do Ceará e do Nordeste;
\item
Impulsionar o desenvolvimento, a produção e a preservação da cultura e das artes, com ênfase para as manifestações regionais.
\item
Promover a interação com a sociedade, através da difusão científica, tecnológica, artística e cultural e do desenvolvimento comunitário, sintonizados com as demandas sociais;
\item
Incentivar a capacitação permanente dos quadros docente e técnico-administrativo;
\item
Intensificar e ampliar as relações de parceria e intercâmbio com instituições nacionais e estrangeiras, governamentais e não governamentais;
\item
Buscar a profissionalização da gestão administrativa, apoiada em processos de planejamento e avaliação, executada com base em modelo organizacional flexível, eficiente e eficaz;
\item
Exercitar permanentemente o instituto da autonomia universitária superando restrições e estabelecendo novos parâmetros na gestão e nas relações institucionais;
\item
Assegurar a qualidade no desenvolvimento de todas as ações administrativas e acadêmicas;
\item
Distinguir-se como referência regional pela excelência acadêmica de suas ações nas áreas do ensino, geração do conhecimento e prestação de serviços à população, bem como na produção de arte e cultura.
\end{enumerate}
\textbf{p. 10}

\subsection{Lema}

“O universal pelo regional” é o lema da UFC, instituição que busca centrar seu compromisso na solução dos problemas locais, sem esquecer o caráter universal de sua produção.
\textbf{p. 11}

\subsection{Missão}

A missão da Universidade é formar profissionais da mais alta qualificação, gerar e difundir conhecimentos, preservar e divulgar os valores éticos, científicos, artísticos e culturais, constituindo-se em instituição estratégica para o desenvolvimento do Ceará, do Nordeste e do Brasil.
\textbf{p. 11}

\subsection{Visão}

Consolidar-se como instituição de referência no ensino de graduação e pós-graduação (stricto e lato sensu), de preservação, geração e produção de ciência e tecnologia, e de integração com o meio, como forma de contribuir para a superação das desigualdades sociais e econômicas, por meio da promoção do desenvolvimento sustentável do Ceará, do Nordeste e do Brasil.
\textbf{p. 11}

\subsection{Centros, faculdades e institutos}

\begin{enumerate}
\item 
Centro de Ciências
\item 
Centro de Humanidades
\item 
Centro de Tecnologia
\item 
Centro de Ciências Agrárias
\item 
Faculdade de Medicina
\item 
Faculdade de Farmácia, Odontologia e Enfermagem
\item 
Faculdade de Direito
\item 
Faculdade de Educação
\item 
Faculdade de Economia, Administração, Atuariais, Contabilidade e Secretariado Executivo
\item 
Instituto de Ciências do Mar (LABOMAR)
\item 
Instituto de Cultura e Arte (ICA)
\item 
Instituto de Educação Física e Esportes (IEFES)
\item 
Instituto Universidade Virtual – UFC Virtual
\end{enumerate}
\textbf{p. 13}

\subsection{Orçamento da UFC - em milhões}
\begin{tabular}{llrrrrrr}
\toprule
{} &           ESPECIFICAÇÃO &    2008 &    2009 &    2010 &    2011 &    2012 &     2013 \\
\midrule
0 &                Previsão &  804.05 &  748.06 &  901.89 &  934.76 &  974.62 &  1116.86 \\
1 &               Executado &  771.74 &  774.84 &  917.41 &  927.40 &  977.95 &  1119.66 \\
2 &  Executado/Previsão (\%) &   95.98 &  103.58 &  101.72 &   99.21 &  100.34 &   100.25 \\
\bottomrule
\end{tabular}

\textbf{p. 56}

\subsection{Área construída da UFC - mil m2}
\begin{tabular}{llrrrrr}
\toprule
{} &             ESPECIFICAÇÃO &    2003 &    2004 &    2005 &    2006 &    2007 \\
\midrule
0 &  Área Construída (mil m2) &  233.63 &  233.63 &  233.63 &  235.14 &  235.14 \\
\bottomrule
\end{tabular}

\begin{tabular}{llrrrrrr}
\toprule
{} &             ESPECIFICAÇÃO &    2008 &    2009 &    2010 &    2011 &    2012 &    2013 \\
\midrule
0 &  Área Construída (mil m2) &  262.73 &  277.48 &  291.31 &  307.57 &  317.23 &  340.67 \\
\bottomrule
\end{tabular}
\textbf{p. 56}


\subsection{Quantitativos de bolsas por ano e modalidade}

\begin{tabular}{llrrrrrrr}
\toprule
{} &                               ESPECIFICAÇÃO &  2007 &  2008 &  2009 &  2010 &  2011 &  2012 &  2013 \\
\midrule
0  &                    Aprendizagem Cooperativa &     - &     - &    98 &   226 &   250 &   250 &   233 \\
1  &                        Apoio Administrativo &     - &     - &     - &   100 &   100 &   200 &   170 \\
2  &                              Cultura e Arte &     - &     - &    64 &    64 &    64 &    80 &   100 \\
3  &                                    Desporto &     - &     - &     - &    30 &    50 &   100 &   100 \\
4  &                                    Extensão &     - &     - &   274 &   378 &   618 &   670 &   650 \\
5  &                Iniciação Científica - PIBIC &   665 &   726 &   782 &   769 &   942 &   925 &   914 \\
6  &                  Iniciação Acadêmica - PRAE &     - &     - &   500 &   580 &   826 &   756 &   900 \\
7  &   Iniciação à Docência - Remunerada - PIBID &     - &     - &   442 &   605 &   700 &   788 &   732 \\
8  &                                 Informática &     - &     - &    95 &    95 &   100 &   100 &   100 \\
9  &            Monitoria de Projeto (Graduação) &     - &     - &   137 &   256 &   300 &   299 &   276 \\
10 &   Programa de Educação Tutorial - PET - UFC &     - &    24 &    76 &   156 &   280 &   288 &   288 \\
11 &  Programa de Educação Tutorial - PET - SESu &     - &   204 &   204 &   204 &   252 &   252 &   252 \\
\bottomrule
\end{tabular}

\textbf{p. 53}

\subsection{Projetos e programas de inclusão social em 2013}
dados não estruturados - esses projetos atuam na dimensão participação/inserção do discente
\textbf{p. 43~44}

\subsection{Unidade Universitária Federal de Educação Infantil - UUFEI}
suporte para discentes que possuam crianças - estranho serem apenas ~60, imaginava mais - p.47 diz serem 54\% filhos de discentes em 2013
\textbf{p. 45~47}

\subsection{Quantitativos de inscritos(vestibular e sisu) e vagas ofertadas por ano}
\begin{tabular}{lrrrr}
\toprule
{} &      2005 &      2006 &      2007 &      2008 \\
\midrule
ESPECIFICAÇÃO        &           &           &           &           \\
Candidatos Inscritos &  42616.00 &  36719.00 &  37771.00 &  31328.00 \\
Vagas Ofertadas      &  3625.00 &  3605.00 &  4045.00 &  4085.00 \\
Demanda              &  11.76 &  10.19 &  9.34 &  7.67 \\
\bottomrule
\end{tabular}

\begin{tabular}{lrrrrr}
\toprule
{} &      2009 &      2010 &     2011 &       2012 &       2013 \\
\midrule
ESPECIFICAÇÃO        &           &           &          &            &            \\
Candidatos Inscritos &  32490.00 &  44156.00 &  78415.0 &  110914.00 &  133923.00 \\
Vagas Ofertadas      &  4484.00 &  5524.00 &  5724.0 &  5834.00 &  6308.00 \\
Demanda              &  7.25 &  7.99 &  13.7 &  19.01 &  21.23 \\
\bottomrule
\end{tabular}

\textbf{p. 49}

\subsection{Quantitativo de cursos de graduação por ano}
\begin{tabular}{lrrrrrrrrrrr}
\toprule
{} &  2003 &  2004 &  2005 &  2006 &  2007 &  2008 &  2009 &  2010 &  2011 &  2012 &  2013 \\
\midrule
ESPECIFICAÇÃO &       &       &       &       &       &       &       &       &       &       &       \\
Nº de Cursos  &  53 &  54 &  54 &  69 &  70 &  73 &  76 &  101 &  105 &  108 &  114 \\
\bottomrule
\end{tabular}

\textbf{p. 50}

\subsection{Indicadores de assistência a estudantes}
\begin{tabular}{llrrrrr}
\toprule
{} &                                          ESPECIFICAÇÃO &  2003 &  2004 &  2005 &  2006 &  2007 \\
\midrule
0 &  Nº de Residências Universitarias & - & - & - & - & - \\
1 &  Nº de Alunos Residentes (Fortaleza e Interior) &  207 &  214 &  229 &  221 &  250 \\
2 &  Nº de Auxílio Residentes (Alunos) & - & - & - & - & - \\
3 &  Nº de Ajuda de Custo (Alunos)-Participação de eventos & - & - & - & - & - \\
4 &  Nº de Bolsas de Iniciação Acadêmica & - & - & - & - & - \\
5 &  Nº de Bolsas de Incentivo ao Desporto & - & - & - & - & - \\
6 &  Nº de Acompanhamento Psicológico & - & - & - & - & - \\
\bottomrule
\end{tabular}
\begin{tabular}{llrrrrrr}
\toprule
{} &                                          ESPECIFICAÇÃO &  2008 &  2009 &  2010 &  2011 &  2012 &  2013 \\
\midrule
0 &  Nº de Residências Universitarias & - &  15 &  16 &  16 &  14 &  11 \\
1 &  Nº de Alunos Residentes (Fortaleza e Interior) &  284 &  288 &  416 &  458 &  535 &  622 \\
2 &  Nº de Auxílio Residentes (Alunos) & - & - &  416 &  458 &  535 &  622 \\
3 &  Nº de Ajuda de Custo (Alunos)-Participação de eventos &  1161 &  2552 &  3648 &  3782 &  3695 &  4273 \\
4 &  Nº de Bolsas de Iniciação Acadêmica & - &  500 &  580 &  826 &  756 &  900 \\
5 &  Nº de Bolsas de Incentivo ao Desporto & - & - &  30 &  50 &  100 &  100 \\
6 &  Nº de Acompanhamento Psicológico & - &  259 &  179 &  120 &  190 &  423 \\
\bottomrule
\end{tabular}

\textbf{p. 52}

\subsection{Distribuição de docentes e técnicos-administrativos}

\begin{tabular}{lrrrrrrrrrrr}
\toprule
{} &  2003 &  2004 &  2005 &  2006 &  2007 &  2008 &  2009 &  2010 &  2011 &  2012 &  2013 \\
\midrule
ESPECIFICAÇÃO          &       &       &       &       &       &       &       &       &       &       &       \\
Docentes               &  1083 &  1093 &  1100 &  1234 &  1749 &  1772 &  1956 &  2005 &  2024 &  2052 &  2152 \\
Técnico-administrativo &  3109 &  3453 &  3409 &  3430 &  3366 &  3458 &  3420 &  3408 &  3466 &  3458 &  3407 \\
\bottomrule
\end{tabular}
\textbf{p. 54}

\subsection{Quantitativo de vagas de casa de cultura}

!notar a diminuição gradual de vagas de 2008 a 2013!

\begin{tabular}{llrrrrr}
\toprule
{} &           ESPECIFICAÇÃO &  2003 &  2004 &  2005 &  2006 &  2007 \\
\midrule
0 &       Teste de Admissão &  2151 &  1719 &  1670 &  1706 &  1550 \\
1 &        Teste de Seleção &  1636 &  1194 &  1344 &  1706 &   572 \\
2 &  Nº Alunos Matriculados &  6197 &  6051 &  5646 &  5796 &  5867 \\
3 &  Nº Alunos Concludentes &  1492 &  1410 &  1308 &  1316 &   945 \\
\bottomrule
\end{tabular}

\begin{tabular}{llrrrrrr}
\toprule
{} &           ESPECIFICAÇÃO &  2008 &  2009 &  2010 &  2011 &  2012 &  2013 \\
\midrule
0 &       Teste de Admissão &  1766 &  1691 &  1268 &  1276 &  1078 &  1124 \\
1 &        Teste de Seleção &  2363 &  2021 &  1949 &  1803 &  1494 &  1407 \\
2 &  Nº Alunos Matriculados &  5452 &  5806 &  5579 &  4892 &  4014 &  3462 \\
3 &  Nº Alunos Concludentes &  1380 &  1447 &  1149 &  1166 &   936 &  1067 \\
\bottomrule
\end{tabular}

\textbf{p. 55}

\subsection{Indicadores de gestão para o TCU}
\cite{indicadores_TCU}
\begin{tabular}{lll}
\toprule
{} &          2008 &          2009 \\
\midrule
COMPONENTES DOS INDICADORES DE DESEMPENHO                   &               &               \\
AE – Aluno Equivalente da UFC                             &  34023.00 &  33557.62 \\
ATI – Aluno em Tempo Integral                               &  21212.00 &  21461.92 \\
AgE – Aluno Equivalente de Graduação                        &  28080.00 &  27074.62 \\
ApgTI – Aluno da Pós-Graduação em Tempo Integral            &  5615.00 &  6075.00 \\
ArTI – Aluno de Residência em Tempo Integral                &  328.00 &  408.00 \\
AgTI – Aluno de Graduação em Tempo Integral                 &  15269.00 &  14978.92 \\
Ag – Aluno de Graduação                                     &  20991.00 &  21289.00 \\
Apg – Aluno de Pós-Graduação                                &  2808.00 &  3038.00 \\
Ar – Aluno de Residência Médica                             &  164.00 &  204.00 \\
Ndi – Alunos Diplomados                                     &  2520.00 &  2481.00 \\
Ni – Alunos Ingressantes                                    &  4822.00 &  4731.00 \\
Custo corrente com HU (inclui 65\% do HU)                   &  444351055.04 &  473411413.49 \\
Custo corrente sem HU                                       &  426930950.49 &  431030343.74 \\
Número de funcionários Equivalente com HU                   &  3313.00 &  3252.50 \\
Número de funcionários Equivalente sem HU                   &  1902.25 &  1916.25 \\
Professor Equivalente                                       &  1619.00 &  1765.50 \\
I.A. Custo corrente com HU/Aluno Equivalente                &  13060.38 &  14107.42 \\
I.B. Custo Corrente sem HU/Aluno Equivalente                &  12548.36 &  12844.49 \\
II. Aluno Tempo Integral/Professor Equivalente              &  13.10 &  12.16 \\
III.A. Aluno Tempo Integral/Funcionário Equivalente com HU  &  6.40 &  6.60 \\
III.B. Aluno Tempo Integral/Funcionário Equivalente sem HU  &  11.15 &  11.20 \\
IV.A. Funcionário Equivalente com HU/Professor Equivalente3 &  2.05 &  1.84 \\
IV.B. Funcionário Equivalente sem HU/Professor Equivalente  &  1.17 &  1.09 \\
V. Grau de Participação Estudantil-GPE                      &  0.73 &  0.70 \\
VI. Grau de Envolvimento com Pós-Graduação-GEPG             &  0.12 &  0.12 \\
VII. Conceito CAPES para a Pós-Graduação                    &  4.13 &  4.11 \\
VIII. Índice de Qualificação do Corpo Docente-IQCD          &  3.95 &  3.73 \\
IX. Taxa de Sucesso na Graduação-TSG                        &  70.00 &  66.86 \\
\bottomrule
\end{tabular}
\newpage
\begin{tabular}{lll}
\toprule
{} &          2010 &          2011 \\
\midrule
COMPONENTES DOS INDICADORES DE DESEMPENHO                   &               &               \\
AE – Aluno Equivalente da UFC                              &  37908.26 &  40708.72 \\
ATI – Aluno em Tempo Integral                               &  23307.93 &  25035.20 \\
AgE – Aluno Equivalente de Graduação                        &  31631.26 &  33018.72 \\
ApgTI – Aluno da Pós-Graduação em Tempo Integral            &  5839.00 &  7308.00 \\
ArTI – Aluno de Residência em Tempo Integral                &  438.00 &  382.00 \\
AgTI – Aluno de Graduação em Tempo Integral                 &  17030.93 &  17345.20 \\
Ag – Aluno de Graduação                                     &  22538.00 &  25971.00 \\
Apg – Aluno de Pós-Graduação                                &  2920.00 &  3654.00 \\
Ar – Aluno de Residência Médica                             &  219.00 &  191.00 \\
Ndi – Alunos Diplomados                                     &  2586.00 &  2792.00 \\
Ni – Alunos Ingressantes                                    &  6204.00 &  5643.00 \\
Custo corrente com HU (inclui 65\% do HU)                   &  564453156.89 &  581255114.03 \\
Custo corrente sem HU                                       &  513713119.26 &  491835392.86 \\
Número de funcionários Equivalente com HU                   &  3255.50 &  3283.25 \\
Número de funcionários Equivalente sem HU                   &  1954.00 &  1927.00 \\
Professor Equivalente                                       &  1856.00 &  1851.50 \\
I.A. Custo corrente com HU/Aluno Equivalente                &  14889.98 &  14278.39 \\
I.B. Custo Corrente sem HU/Aluno Equivalente                &  13551.48 &  12081.82 \\
II. Aluno Tempo Integral/Professor Equivalente              &  12.56 &  13.52 \\
III.A. Aluno Tempo Integral/Funcionário Equivalente com HU  &  7.16 &  7.63 \\
III.B. Aluno Tempo Integral/Funcionário Equivalente sem HU  &  11.93 &  12.99 \\
IV.A. Funcionário Equivalente com HU/Professor Equivalente3 &  1.75 &  1.77 \\
IV.B. Funcionário Equivalente sem HU/Professor Equivalente  &  1.05 &  1.04 \\
V. Grau de Participação Estudantil-GPE                      &  0.76 &  0.67 \\
VI. Grau de Envolvimento com Pós-Graduação-GEPG             &  0.11 &  0.12 \\
VII. Conceito CAPES para a Pós-Graduação                    &  4.22 &  4.22 \\
VIII. Índice de Qualificação do Corpo Docente-IQCD          &  4.03 &  4.13 \\
IX. Taxa de Sucesso na Graduação-TSG                        &  68.45 &  69.06 \\
\bottomrule
\end{tabular}
\newpage
\begin{tabular}{lll}
\toprule
{} &          2012 &          2013 \\
\midrule
COMPONENTES DOS INDICADORES DE DESEMPENHO                   &               &               \\
AE – Aluno Equivalente da UFC                              &  41144.35 &  42443.53 \\
ATI – Aluno em Tempo Integral                               &  26330.48 &  26466.34 \\
AgE – Aluno Equivalente de Graduação                        &  32468.35 &  34247.53 \\
ApgTI – Aluno da Pós-Graduação em Tempo Integral            &  8268.00 &  7760.00 \\
ArTI – Aluno de Residência em Tempo Integral                &  408.00 &  436.00 \\
AgTI – Aluno de Graduação em Tempo Integral                 &  17654.48 &  18270.34 \\
Ag – Aluno de Graduação                                     &  26956.00 &  27433.00 \\
Apg – Aluno de Pós-Graduação                                &  4134.00 &  3880.00 \\
Ar – Aluno de Residência Médica                             &  204.00 &  218.00 \\
Ndi – Alunos Diplomados                                     &  2684.00 &  2920.00 \\
Ni – Alunos Ingressantes                                    &  6406.00 &  6087.00 \\
Custo corrente com HU (inclui 65\% do HU)                   &  560737712.22 &  698496687.71 \\
Custo corrente sem HU                                       &  482034252.71 &  609763905.54 \\
Número de funcionários Equivalente com HU                   &  3281.50 &  3277.75 \\
Número de funcionários Equivalente sem HU                   &  1990.00 &  2047.50 \\
Professor Equivalente                                       &  1912.50 &  1948.50 \\
I.A. Custo corrente com HU/Aluno Equivalente                &  13628.55 &  16457.08 \\
I.B. Custo Corrente sem HU/Aluno Equivalente                &  11715.69 &  14366.47 \\
II. Aluno Tempo Integral/Professor Equivalente              &  13.77 &  13.58 \\
III.A. Aluno Tempo Integral/Funcionário Equivalente com HU  &  8.03 &  8.07 \\
III.B. Aluno Tempo Integral/Funcionário Equivalente sem HU  &  13.23 &  12.93 \\
IV.A. Funcionário Equivalente com HU/Professor Equivalente3 &  1.72 &  1.68 \\
IV.B. Funcionário Equivalente sem HU/Professor Equivalente  &  1.04 &  1.05 \\
V. Grau de Participação Estudantil-GPE                      &  0.65 &  0.67 \\
VI. Grau de Envolvimento com Pós-Graduação-GEPG             &  0.13 &  0.12 \\
VII. Conceito CAPES para a Pós-Graduação                    &  4.20 &  4.34 \\
VIII. Índice de Qualificação do Corpo Docente-IQCD          &  4.15 &  4.24 \\
IX. Taxa de Sucesso na Graduação-TSG                        &  66.63 &  56.51 \\
\bottomrule
\end{tabular}
\textbf{p. 57~58}

\subsection{Porcentagem de alunos de graduação com bolsas}
\begin{tabular}{llrrrrrr}
\toprule
{} &                                   ESPECIFICAÇÃO &  2008 &  2009 &  2010 &  2011 &  2012 &  2013 \\
\midrule
0 &  Iniciação Científica - PIBIC e PET(Sesu e UFC) &   4.5 &   4.6 &   4.7 &   5.2 &   5.4 &  5.43 \\
1 &                              Bolsa de Monitoria &   5.2 &   5.7 &   7.7 &   7.9 &   7.2 &  6.81 \\
\bottomrule
\end{tabular}

\textbf{p. 53}

\subsection{Bolsas de monitoria, pet - por curso em 2013; de ic - por área de conhecimento 2010 a 2013}
\textbf{p. 282 a 289}

\section{Evasão}
\cite{esclarecimentos_calculos}
\cite{mudanca_calculos}

\subsection{Definição}
Por evasão de discente entenda-se o abandono, pelo discente, de um processo de aprendizado antes de sua conclusão. Por exemplo, um discente que abandonou o curso de Medicina, na UFC, no qual estava matriculado pois precisou trabalhar para sustentar sua família e os horários das disciplinas eram incompatíveis com os horários do trabalho. Especificando-se o escopo do processo de aprendizado temos definições mais objetivas de evasão de discente. Por exemplo:
\begin{itemize}
\item Evasão de curso: refere-se ao abandono, pelo discente, de um curso, antes de sua conclusão.
\item Evasão de área de conhecimento: refere-se ao abandono, pelo discente, de um curso, antes de sua conclusão, sem posterior ingresso em curso da mesma área de conhecimento.
\item Evasão de IES: refere-se ao abandono, pelo discente, de um curso, antes de sua conclusão, sem posterior ingresso em curso da mesma IES. 
\item Evasão do ensino: refere-se ao abandono, pelo discente, de um curso, antes de sua conclusão, sem posterior ingresso em outro curso.
\end{itemize}

Esses exemplos não esgotam as possibilidades de definições de evasão de discente, visto ser bastante flexível a definição do escopo do processo de aprendizado. A escolha do escopo a ser considerado deve ser direcionada pelo uso que será feito dos dados calculados a partir da definição de evasão de discente. Por exemplo, para fins de planejamento de políticas de fomento de uma área de conhecimento, pode-se utilizar a definição Evasão de área de conhecimento a fim de verificar se está havendo migração de discentes da área de conhecimento em foco para outras; para fins de comparação, os valores devem ser gerados seguindo a mesma definição, como no caso entre comparações de índices de Evasão de IES entre as IES.

De acordo com a literatura analisada \cite{tinto_leaving} \cite{evasao_panorama} a quantidade de evasão da IES ocorrem no primeiro ano do discente na instituição.

A evasão de discente configura-se em problema ao considerarmos que o custo x benefício do processo de aprendizado, quando abandonado antes da conclusão, é menor que o esperado. \todo{expressão de problema da evasão de discentes}Expectativas podem ser frustradas, como a esperança de formação de corpo técnico para atender às demandas da indústria; recursos podem ficar ociosos, como é o caso de contratação de servidores, a aquisição de equipamentos e a construção de espaços físicos; pode haver diminuição de recursos financeiros da instituição, caso seu orçamento seja vinculado à quantidade de discentes que concluem o processo de aprendizado. O cálculo 

\subsection{Indicadores de evasão}

\subsubsection{Taxa de sucesso}

Definido com esse nome em \cite{indicadores_TCU} e com o nome "Taxa de titulação" em \cite{mudanca_calculos}, esse indicador é a razão entre a quantidade de discentes concluintes efetivos e a quantidade de concluintes esperados, considerando prazo esperado de conclusão.
Este indicador abstrai a ocorrência de ocupação, via transferência, de vaga ociosa por evasão; considera apenas as conclusões no prazo esperado, podendo gerar dados inesperados, como percentuais acima de 100\%, indicando haver se formado no ano em questão mais discentes que o esperado, consequência, por exemplo, de discentes atrasados ou de discentes adiantados.
\begin{equation}
\frac{Nº\ de\ diplomados(N_{DI})}{Nº\ total\ de\ alunos\ ingressantes}
\end{equation}

\subsubsection{Evasão anual}

Definido em \cite{esclarecimentos_calculos} e \cite{mudanca_calculos}, é a razão entre a quantidade de rematrículas realizadas e a quantidade de rematrículas esperadas.
\begin{equation}
1 - \frac{M(n) - Ig(n)}{M(n-1) - Eg(n-1)}
\end{equation}
\begin{itemize}
\item $M(n)$: quantidade de matrículas na unidade de tempo $n$
\item $Eg(n)$: quantidade de egressos na undiade de tempo $n$
\item $Ig(n)$: quantidade de ingressantes na unidade de tempo $n$
\end{itemize}

Para derivar essa taxa, basta notarmos as seguintes equações e respectivos significados:
\begin{enumerate}
\item 
\begin{equation}
M(n) = E(n) + X(n) + Eg(n)
\end{equation}
significando que a quantidade de matrículas em determinada unidade de tempo $n$ é composta pelas quantidades de matrículas de discentes que evadirão em $n$, de discentes que persistirão, isto é, se matricularão em $n+1$, e de discentes que se diplomarão ao fim de $n$.
\item 
\begin{equation}
M(n) = X(n-1) + Ig(n)
\end{equation}
significando que a quantidade de matrículas em determinada unidade de tempo $n$ é composta pelas quantidades de matrículas de discentes que se matricularam em $n$ e em $n-1$, ou seja, que persistiram de $n-1$ a $n$, e de discentes que ingressaram em $n$.
\end{enumerate}

Fazendo as devidas manipulações nas equações, chegamos na equação de quantidade de evasões ao fim de determinada unidade de tempo:
\begin{equation}
E(n) = [M(n) - Eg(n)] - [M(n+1) - Ig(n+1)]
\end{equation}
Podendo ser interpretada como a diferença de quantidade máxima de discentes persistindo e a quantidade efetiva de discentes que persistiram.

A fórmula da evasão anual pode ser então definida como a proporção entre a quantidade de evasões em uma unidade de tempo e a quantidade máxima de discentes persistindo naquela unidade de tempo.

A definição de evasão anual até então analisada refere-se à evasão de cursos, de forma que um discente que tenha mudado de curso na mesma IES terá ao mesmo tempo contabilizados sua matrícula no período anterior($M(n-1)$) e como ingressante($Ig(n)$).

\todo{analisar o ciclo de vida dos discentes e mapear esses números em size de sets}
Para a evasão de IES, é proposta a seguinte definição:
\begin{equation}
1 - \frac{M(n) - [Ig(n) - ITC(n)]}{M(n-1) - Eg(n-1)}
\end{equation}
\begin{itemize}
\item $ITC(n)$: quantidade de ingressantes no ano $n$ com forma de ingresso transferência interna
\end{itemize}

Já para a evasão do sistema, a ser aplicada a dados de várias, se possível todas, IES, a definição proposta é:
\begin{equation}
1 - \frac{M(n) - [Ig(n) - ITC(n) - ITIES(n)]}{M(n-1) - Eg(n-1)}
\end{equation}
\begin{itemize}
\item $ITIES(n)$: quantidade de ingressantes no ano $n$ com forma de ingresso transferência externa
\end{itemize}

\todo[inline]{refletir melhor sobre o significado desses indicadores}
\todo[inline]{refletir melhor sobre quais possíveis indicadores utilizar}
\todo[inline]{Analisar os indicadores de evasão da literatura e calcular}

\subsection{Causas}

\subsubsection{Instituto Lobo}
De acordo com o Instituto Lobo\cite{evasao_panorama2}, as causas mais comuns de evasão do sistema de ensino superior no Brasil são:

\begin{enumerate}

\item \textbf{Baixa qualidade da educação básica brasileira}: mensurada pelos exames internacionais\todo{quais?}.

\item \textbf{Baixa eficiência e o diploma do ensino médio}: que não garante a suficiência de competência do candidato ao ensino superior, criando dificuldades de adaptação e acompanhamento do curso.

\item \textbf{Limitação das políticas de financiamento ao estudante}: mesmo com FIES e PROUNI são insuficientes\todo{pq?}, e para alunos do setor público que, em certos casos, deixam de estudar por não terem meios financeiros para se manterem.

\item \textbf{Escolha precoce da especialidade profissional}: incorre aqui a existência de grande quantidade de cursos de graduação(mais de 200), muito especializados. A título de exemplo, o curso de Direito no Brasil tem duração de 5 anos e ao fim já garante o exercício profissional ao formado após o exame da Ordem, quando em outros países esse curso é uma espécie de pós-graduação.

\item \textbf{Dificuldade de mobilidade estudantil}: composta pela dificuldade de transferência entre IES e pela dificuldade de aproveitamento de créditos. A tendência em países desenvolvidos é a unificação de currículos, que diminui tais dificuldades\todo{citado Processo de Bolonha}.

\item \textbf{Rigidez do arcabouço legal e das exigências para autorização/reconhecimento de cursos}: há um risco de inovações nos projetos pedagógicos não serem aprovadas pela Comissão de autorização e/ou reconhecimento.

\item \textbf{Falta de pressão para combater a evasão}: em virtude da cultura acadêmica, pela qual um curso nasce e responde às necessidades e visão dos docentes, em especial das IES públicas(e até de sindicados, associações de classe e profissionais que trabalham muitas vezes pela reserva de mercado e manutenção do status quo).\todo{copiei - problema sério a visão ser academicista e não voltada ao discente}

\item \textbf{Legislação sobre inadimplência no Brasil}: uma excrecência demagógica que educa para o calote, favorece o acúmulo de dívidas pelo aluno e a Evasão das IES privadas.\todo{copiei - desconheço o assunto}

\item \textbf{Enorme quantidade de docentes despreparados para o ensino e para lidar com o aluno real}: o que ocorre, entre muitas razões, pela falta de formação didático-pedagógica de vários deles e pela acomodação oriunda da estabilidade precoce de muitos(por força legal nas IES públicas e de fato nas IES privadas), tudo isso somado à dificuldade de cobrança de desempenho e à pequena valorização do ensino nos planos e promoções de carreira docente, com valorização quase exclusiva da produção científica\todo{copiei}

\item \textbf{Inadaptação ao estilo do ensino superior}

\item \textbf{Dificuldade financeira}

\item \textbf{Irritação com a precariedade dos serviços oferecidos pela IES}

\item \textbf{Decepção com a pouca motivação e atenção dos professores}

\item \textbf{Dificuldades com transporte, alimentação e ambientação na IES}

\item \textbf{Mudança de curso}

\item \textbf{Mudança de residência}

\end{enumerate}

Em \cite{evasao_panorama2} são apresentados modelos teóricos que tentam explicar a evasão e são precursores do trabalho de Vincent Tinto, indicado com um dos maiores especialistas em evasão.

\begin{itemize}
\item \textbf{Modelos psicológicos}: Belief, attitude, intention and behavior: An introduction to theory and research(Ajzen 1975); A psychological model of student persistence(Ethington 1990).

\item \textbf{Modelos de integração estudante - instituição}: baseia-se no fato de que a integração do aluno com a IES é fundamental para a sua permanência.\todo{é citado Spady 1975, mas não achei a referência...}
\end{itemize}

É citada frase de T.E.Corts, ex-presidente da Samford University, que copio em parte:
\begin{quote}
Algumas pesquisas indicam que estudantes não abandonam faculdades por grandes razões, mas pelo acúmulo de pequenas razões que destroem suas justificativas de escolha de uma instituição.
\end{quote}

\subsubsection{Opinião de docentes e de coordenadores acerca do fenômeno da evasão discente dos cursos de graduação da Universidade Federal do Ceará(UFC)}

No trabalho \cite{andriola} são apresentados os resultados de pesquisa feita com uma amostra(tamanho 86 de um universo de tamanho 412) de discentes evadidos no período de 1999 e 2000 acerca dos motivos que os levaram a abandonar os cursos. Os seguintes fatores foram apresentados:

\begin{enumerate}

\item Incompatibilidade entre horários de trabalho e de estudo(destacado por 39.4\% dos evadidos).

\item Aspectos familiares(por exemplo: necessidade de dedicar-se aos filhos menores) e desmotivação com os estudos(justificado por 20\% dos evadidos).

\item Precariedade das condições físicas do curso ou inadequação curricular(mencionado por 10\% dos evadidos).

\end{enumerate}


\subsubsection{Vincent Tinto}

Em \cite{evasao_panorama2} é citada a crítica de Tinto à formação didático-pedagógica dos professores do ensino superior; sua ênfase na interação do aluno na IES; seu alerta para que sejam analisadas as causas em todos os envolvidos no processo, e não apenas no aluno; a maior concentração de evasão no primeiro ano; a preocupação com a diminuição na qualidade da educação dos ingressantes, recomendando a adoção de programas específicos para os perfis dos novos aunos; e o engano que pode ser causado por aceitarmos a justificativa de problema financeiro para a evasão, quando outras causa podem ter tido maior influência.

\todo[inline]{Colaboração de Leaving College}

\subsection{Soluções}

\subsubsection{Instituto Lobo}

Em \cite{evasao_panorama2} são apresentados "Sete pontos para baixar a evasão":

\begin{enumerate}

\item Estabelecer um grupo de trabalho encarregado de reduzir a evasão.

\item Avaliar as estatísticas da evasão.

\item Determinar as causas da evasão.

\item Estimular a visão da IES centrada no aluno.

\item Criar condições que atendam aos objetivos que atraíram os alunos.

\item Tornar o ambiente e o trânsito na IES agradáveis aos alunos.

\item Criar programa de aconselhamento e orientação dos alunos.

\end{enumerate}

E "Conclusões e recomendações sobre a evasão no ensino superior brasileiro":

\begin{enumerate}

\item O problema da evasão deve ser discutido com todos os envolvidos no processo.

\item O problema da evasão deve ser combatido em todos os niveis, não apenas no operacional.

\item São necessários dados confiáveis e organizados para se realizar bons planejamentos.

\item É necessário buscar a integração das áreas acadêmicas e administrativo-financeira das IES para que ambas caminhem juntas no combate à evasão.

\item A adoção de equipe técnica para estudar e acompanhar o fenômeno deve ser acompanhada do apoio e do trabalho de todos da instituição.

\item "O comprometimento com o sucesso do aluno implica na coragem de buscar medidas, nem sempre simpáticas aos professores e alunos, para que se garanta o aprendizado e a medição desse aprendizado, tais como provas elaboradas por outros professores, avaliações de desempenho com consequências etc."

\item "Afirmar que as questões financeiras(da IES e do aluno) não dizem respeito à academia, é ignorar que tudo o que afeta a missão de uma IES envolve, necessariamente, a academia."

\item "No fundo, todos os problemas de uma IES passam, necessariamente, pela GESTÃO. A gestão universitária é uma profissão para a qual é preciso treinar os professores e profissionais que a ela se dedicam, ou pensam/se propõem a se dedicar e, por isso, os gestores das IES precisam ser capacitados para entender e combater a evasão"

\item "Não se pode ensinar ao aluno a ser um profissional e um cidadão comprometido quando uma IES demonstra amadorismo em seus processos e descuidos em relação ao seu maior compromisso: o auno, que é a razão de ser de uma IES!"

\end{enumerate}

\subsubsection{Vincent Tinto}
\todo[inline]{soluções apontadas por Vincent Tinto}

\subsubsection{Opinião de docentes e de coordenadores acerca do fenômeno da evasão discente dos cursos de graduação da Universidade Federal do Ceará(UFC)}
\todo[inline]{sugestões de docentes e coordenadores}

\chapter{Business objectives}

\chapter{Business success criteria}

\cite{anuario_2014_base_2013}
\cite{pdi_ufc}

\bibliographystyle{acm}
\bibliography{../../../referencias}

\end{document}
