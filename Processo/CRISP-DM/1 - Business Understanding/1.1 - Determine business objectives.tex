\documentclass{report}
\usepackage{url}
\usepackage{todonotes}
\usepackage{booktabs}
\usepackage[utf8]{inputenc}	

\begin{document}

\listoftodos

\tableofcontents

\chapter{Background}

\section{Evasão}
\cite{esclarecimentos_calculos}
\cite{mudanca_calculos}

\subsection{Definição}
Por evasão de discente entenda-se a interrupção de um processo de aprendizado de um discente antes de sua conclusão. Por exemplo, um discente que abandonou o curso de Medicina, na UFC, no qual estava matriculado pois precisou trabalhar para sustentar sua família e os horários das disciplinas eram incompatíveis com os horários do trabalho. Especificando-se o escopo do processo de aprendizado temos definições mais objetivas de evasão de discente. Por exemplo:
\begin{itemize}
\item Evasão de curso: refere-se ao abandono, pelo discente, de um curso, antes de sua conclusão.
\item Evasão de área de conhecimento: refere-se ao abandono, pelo discente, de um curso, antes de sua conclusão, sem posterior ingresso em curso da mesma área de conhecimento.
\item Evasão de IES: refere-se ao abandono, pelo discente, de um curso, antes de sua conclusão, sem posterior ingresso em curso da mesma IES. 
\item Evasão do ensino: refere-se ao abandono, pelo discente, de um curso, antes de sua conclusão, sem posterior ingresso em outro curso.
\end{itemize}

Esses exemplos não esgotam as possibilidades de definições de evasão de discente, visto ser bastante flexível a definição do escopo do processo de aprendizado. A escolha do escopo a ser considerado deve ser direcionada pelo uso que será feito dos dados calculados a partir da definição de evasão de discente. Por exemplo, para fins de planejamento de políticas de fomento de uma área de conhecimento, pode-se utilizar a definição Evasão de área de conhecimento a fim de verificar se está havendo migração de discentes da área de conhecimento em foco para outras; para fins de comparação, os valores devem ser gerados seguindo a mesma definição, como no caso entre comparações de índices de Evasão de IES entre as IES.

Além do escopo, outro atributo da evasão que pode ser levado em consideração é o agente cuja decisão resultou na interrupção do processo de aprendizado, podendo ser tanto o discente quanto a instituição de ensino.

\todo[inline]{discutir o atributo 'motivo' da evasão}
\todo[inline]{discutir o atributo 'pós evasão'}

De acordo com a literatura analisada \cite{tinto_leaving} \cite{evasao_panorama} a maior parte da evasão de IES ocorre no primeiro ano do discente na instituição.

A evasão de discente configura-se em problema ao considerarmos que o custo x benefício do processo de aprendizado, quando abandonado antes da conclusão, é menor que o esperado. Expectativas podem ser frustradas, como a esperança de formação de corpo técnico para atender às demandas da indústria; recursos podem ficar ociosos, como é o caso de contratação de servidores, a aquisição de equipamentos e a construção de espaços físicos; pode haver diminuição de recursos financeiros da instituição, caso seu orçamento seja vinculado à quantidade de discentes que concluem o processo de aprendizado. 

\subsection{Indicadores de evasão}

\subsubsection{Taxa de sucesso}

Definido com esse nome em \cite{indicadores_TCU} e com o nome "Taxa de titulação" em \cite{mudanca_calculos}, esse indicador é a razão entre a quantidade de discentes concluintes efetivos e a quantidade de concluintes esperados, considerando prazo esperado de conclusão.
Este indicador abstrai a ocorrência de ocupação, via transferência, de vaga ociosa por evasão; considera apenas as conclusões no prazo esperado, podendo gerar dados inesperados, como percentuais acima de 100\%, indicando haver se formado no ano em questão mais discentes que o esperado, consequência, por exemplo, de discentes atrasados ou de discentes adiantados.
\begin{equation}
\frac{Nº\ de\ diplomados(N_{DI})}{Nº\ total\ de\ alunos\ ingressantes}
\end{equation}

\todo[inline]{discutir as vantagens e desvantagens dessa métrica e como interpretá-la}

\subsubsection{Evasão anual}

Definido em \cite{esclarecimentos_calculos} e \cite{mudanca_calculos}, é a razão entre a quantidade de rematrículas realizadas e a quantidade de rematrículas esperadas.
\begin{equation}
1 - \frac{M(n) - Ig(n)}{M(n-1) - Eg(n-1)}
\end{equation}
\begin{itemize}
\item $M(n)$: quantidade de matrículas na unidade de tempo $n$
\item $Eg(n)$: quantidade de egressos na undiade de tempo $n$
\item $Ig(n)$: quantidade de ingressantes na unidade de tempo $n$
\end{itemize}

Para derivar essa taxa, basta notarmos as seguintes equações e respectivos significados:
\begin{enumerate}
\item 
\begin{equation}
M(n) = E(n) + X(n) + Eg(n)
\end{equation}
significando que a quantidade de matrículas em determinada unidade de tempo $n$ é composta pelas quantidades de matrículas de discentes que evadirão em $n$, de discentes que persistirão, isto é, se matricularão em $n+1$, e de discentes que se diplomarão ao fim de $n$.
\item 
\begin{equation}
M(n) = X(n-1) + Ig(n)
\end{equation}
significando que a quantidade de matrículas em determinada unidade de tempo $n$ é composta pelas quantidades de matrículas de discentes que se matricularam em $n$ e em $n-1$, ou seja, que persistiram de $n-1$ a $n$, e de discentes que ingressaram em $n$.
\end{enumerate}

Fazendo as devidas manipulações nas equações, chegamos na equação de quantidade de evasões ao fim de determinada unidade de tempo:
\begin{equation}
E(n) = [M(n) - Eg(n)] - [M(n+1) - Ig(n+1)]
\end{equation}
Podendo ser interpretada como a diferença de quantidade máxima de discentes persistindo e a quantidade efetiva de discentes que persistiram.

A fórmula da evasão anual pode ser então definida como a proporção entre a quantidade de evasões em uma unidade de tempo e a quantidade máxima de discentes persistindo naquela unidade de tempo.

A definição de evasão anual até então analisada refere-se à evasão de cursos, de forma que um discente que tenha mudado de curso na mesma IES terá ao mesmo tempo contabilizados sua matrícula no período anterior($M(n-1)$) e como ingressante($Ig(n)$).

\todo{analisar o ciclo de vida de discente e derivar os indicadores}
Para a evasão anual de IES, é proposta a seguinte definição:
\begin{equation}
1 - \frac{M(n) - [Ig(n) - ITC(n)]}{M(n-1) - Eg(n-1)}
\end{equation}
\begin{itemize}
\item $ITC(n)$: quantidade de ingressantes no ano $n$ com forma de ingresso transferência interna
\end{itemize}

Já para a evasão anual do sistema, a ser aplicada a dados de várias, se possível todas, IES, a definição proposta é:
\begin{equation}
1 - \frac{M(n) - [Ig(n) - ITC(n) - ITIES(n)]}{M(n-1) - Eg(n-1)}
\end{equation}
\begin{itemize}
\item $ITIES(n)$: quantidade de ingressantes no ano $n$ com forma de ingresso transferência externa
\end{itemize}

\todo[inline]{discutir as vantagens e desvantagens dessa métrica e como interpretá-la}
\todo[inline]{calcular indicadores para a UFC}

\subsection{Causas}

\subsubsection{Fonte: Instituto Lobo}
De acordo com o Instituto Lobo\cite{evasao_panorama2}, as causas mais comuns de evasão do sistema de ensino superior no Brasil são:

\begin{enumerate}

\item \textbf{Baixa qualidade da educação básica brasileira}: mensurada pelos exames internacionais\todo{pesquisar indicadores de qualidade da educação básica brasileira}.

\item \textbf{Baixa eficiência e o diploma do ensino médio}: que não garante a suficiência de competência do candidato ao ensino superior, criando dificuldades de adaptação e acompanhamento do curso.

\item \textbf{Limitação das políticas de financiamento ao estudante}: mesmo com FIES e PROUNI são insuficientes\todo{pesquisar discussão sobre limitações nas políticas de financiamento}, e para alunos do setor público que, em certos casos, deixam de estudar por não terem meios financeiros para se manterem.

\item \textbf{Escolha precoce da especialidade profissional}: incorre aqui a existência de grande quantidade de cursos de graduação(mais de 200), muito especializados. A título de exemplo, o curso de Direito no Brasil tem duração de 5 anos e ao fim já garante o exercício profissional ao formado após o exame da Ordem, quando em outros países esse curso é uma espécie de pós-graduação.

\item \textbf{Dificuldade de mobilidade estudantil}: composta pela dificuldade de transferência entre IES e pela dificuldade de aproveitamento de créditos. A tendência em países desenvolvidos é a unificação de currículos, que diminui tais dificuldades.

\item \textbf{Rigidez do arcabouço legal e das exigências para autorização/reconhecimento de cursos}: há um risco de inovações nos projetos pedagógicos não serem aprovadas pela Comissão de autorização e/ou reconhecimento.

\item \textbf{Falta de pressão para combater a evasão}: em virtude da cultura acadêmica, pela qual um curso nasce e responde às necessidades e visão dos docentes, em especial das IES públicas(e até de sindicados, associações de classe e profissionais que trabalham muitas vezes pela reserva de mercado e manutenção do status quo).

\item \textbf{Legislação sobre inadimplência no Brasil}: uma excrecência demagógica que educa para o calote, favorece o acúmulo de dívidas pelo aluno e a Evasão das IES privadas.

\item \textbf{Enorme quantidade de docentes despreparados para o ensino e para lidar com o aluno real}: o que ocorre, entre muitas razões, pela falta de formação didático-pedagógica de vários deles e pela acomodação oriunda da estabilidade precoce de muitos(por força legal nas IES públicas e de fato nas IES privadas), tudo isso somado à dificuldade de cobrança de desempenho e à pequena valorização do ensino nos planos e promoções de carreira docente, com valorização quase exclusiva da produção científica

\item \textbf{Inadaptação ao estilo do ensino superior}

\item \textbf{Dificuldade financeira}

\item \textbf{Irritação com a precariedade dos serviços oferecidos pela IES}

\item \textbf{Decepção com a pouca motivação e atenção dos professores}

\item \textbf{Dificuldades com transporte, alimentação e ambientação na IES}

\item \textbf{Mudança de curso}

\item \textbf{Mudança de residência}

\end{enumerate}

Em \cite{evasao_panorama2} são apresentados modelos teóricos que tentam explicar a evasão e são precursores do trabalho de Vincent Tinto, indicado com um dos maiores especialistas em evasão.

\begin{itemize}
\item \textbf{Modelos psicológicos}: Belief, attitude, intention and behavior: An introduction to theory and research(Ajzen 1975); A psychological model of student persistence(Ethington 1990).

\item \textbf{Modelos de integração estudante - instituição}: baseia-se no fato de que a integração do aluno com a IES é fundamental para a sua permanência.\todo{é citado Spady 1975, mas não achei a referência...}
\end{itemize}

É citada frase de T.E.Corts, ex-presidente da Samford University, que copio em parte:
\begin{quote}
Algumas pesquisas indicam que estudantes não abandonam faculdades por grandes razões, mas pelo acúmulo de pequenas razões que destroem suas justificativas de escolha de uma instituição.
\end{quote}

\subsubsection{Fonte: Opinião de docentes e de coordenadores acerca do fenômeno da evasão discente dos cursos de graduação da Universidade Federal do Ceará(UFC)}

No trabalho \cite{andriola} são apresentados os \textbf{resultados de pesquisa feita com uma amostra(tamanho 86 de um universo de tamanho 412) de discentes evadidos no período de 1999 e 2000 acerca dos motivos que os levaram a abandonar os cursos}. Os seguintes fatores foram apresentados:

\begin{enumerate}

\item Incompatibilidade entre horários de trabalho e de estudo(destacado por 39.4\% dos evadidos).

\item Aspectos familiares(por exemplo: necessidade de dedicar-se aos filhos menores) e desmotivação com os estudos(justificado por 20\% dos evadidos).

\item Precariedade das condições físicas do curso ou inadequação curricular(mencionado por 10\% dos evadidos).

\end{enumerate}


\subsubsection{Vincent Tinto}

Em \cite{evasao_panorama2} é citada a crítica de Tinto à formação didático-pedagógica dos professores do ensino superior; sua ênfase na interação do aluno na IES; seu alerta para que sejam analisadas as causas em todos os envolvidos no processo, e não apenas no aluno; a maior concentração de evasão no primeiro ano; a preocupação com a diminuição na qualidade da educação dos ingressantes, recomendando a adoção de programas específicos para os perfis dos novos aunos; e o engano que pode ser causado por aceitarmos a justificativa de problema financeiro para a evasão, quando outras causa podem ter tido maior influência.

Em \cite{tinto_leaving} Vincent Tinto explora o problema da evasão de discentes primeiro analisando o perfil dos discentes ingressantes e evadidos, depois analisando estudos sobre o fenômeno e desenvolvendo sua própria teoria, finalizando com uma discussão sobre as ações que as instituições podem executar para reduzir o problema.

\textbf{p. 1}

Dos 2.4 milhões de ingressantes em 1993, 1.5 milhões abandonarão a primeira instituição de ingresso, sem se diplomarem; desses, 1.1 milhões não retornarão ao ensino superior.

Relação entre diploma e reward - apresenta o salário anual médio, em 1989, de um profissional com e sem diploma, apontando ser relevante a diferença, mas alertando para que isso não quer dizer que não há benefícios caso o processo seja interrompido antes da conclusão(conhecimento foi adquirido, por exemplo).

\textbf{p. 2}

Apresenta uma informação que também aparece nos documentos do Instituto Lobo: que a oferta e ocupação de vagas, em determinado momento, saturou, fazendo com que o foco das ações das instituições para geração de renda passasse do marketing para aumento de ingressos para aquelas que buscassem aumentar a retenção dos discentes já ingressos.

\textbf{p. 3}

Alerta para que o problema é complexo.

Alerta para que a evasão, do ponto de vista dos discentes, nem sempre é vista como algo negativo. Lembrar que o processo não concluído não implica em ausência de benefícios; além do caso de escolha errada de curso(vocação).

Alerta para a dificuldade em isolar os fatores que contribuem para a evasão e a retenção, de forma que é complicado replicar resultados entre instituições.

\textbf{p. 4}

Alerta, novamente, que os fatores envolvidos na evasão/retenção estão distribuídos no contexto e não concentrados em poucas entidades.

\textbf{p. 5}

Alerta para que, antes de serem executadas ações relacionadas a evasão, é importante que a instituição analise quais os objetivos busca alcançar com tais ações, principalmente em vista das diversas definições de evasão(quer diminuir a evasão da instituição? de curso? transferência por falta de vocação? etc).

A estrutura do livro segue:

\begin{enumerate}

\item ch02: analisa o perfil dos ingressantes e evadidos

\item ch03: analisa estudos até então desenvolvidos
 
\item ch04: desenvolve uma teoria para o fenômeno

\item ch05: discute ações que podem ser executadas para reduzir o problema

\end{enumerate}

\todo[inline]{continuar resumo do Leaving College}

\subsection{Soluções}

\subsubsection{Fonte: Instituto Lobo}

Em \cite{evasao_panorama2} são apresentados "Sete pontos para baixar a evasão":

\begin{enumerate}

\item Estabelecer um grupo de trabalho encarregado de reduzir a evasão.

\item Avaliar as estatísticas da evasão.

\item Determinar as causas da evasão.

\item Estimular a visão da IES centrada no aluno.

\item Criar condições que atendam aos objetivos que atraíram os alunos.

\item Tornar o ambiente e o trânsito na IES agradáveis aos alunos.

\item Criar programa de aconselhamento e orientação dos alunos.

\end{enumerate}

E na seção "Conclusões e recomendações sobre a evasão no ensino superior brasileiro":

\begin{enumerate}

\item O problema da evasão deve ser discutido com todos os envolvidos no processo.

\item O problema da evasão deve ser combatido em todos os niveis, não apenas no operacional.

\item São necessários dados confiáveis e organizados para se realizar bons planejamentos.

\item É necessário buscar a integração das áreas acadêmicas e administrativo-financeira das IES para que ambas caminhem juntas no combate à evasão.

\item A adoção de equipe técnica para estudar e acompanhar o fenômeno deve ser acompanhada do apoio e do trabalho de todos da instituição.

\item "O comprometimento com o sucesso do aluno implica na coragem de buscar medidas, nem sempre simpáticas aos professores e alunos, para que se garanta o aprendizado e a medição desse aprendizado, tais como provas elaboradas por outros professores, avaliações de desempenho com consequências etc."

\item "Afirmar que as questões financeiras(da IES e do aluno) não dizem respeito à academia, é ignorar que tudo o que afeta a missão de uma IES envolve, necessariamente, a academia."

\item "No fundo, todos os problemas de uma IES passam, necessariamente, pela GESTÃO. A gestão universitária é uma profissão para a qual é preciso treinar os professores e profissionais que a ela se dedicam, ou pensam/se propõem a se dedicar e, por isso, os gestores das IES precisam ser capacitados para entender e combater a evasão"

\item "Não se pode ensinar ao aluno a ser um profissional e um cidadão comprometido quando uma IES demonstra amadorismo em seus processos e descuidos em relação ao seu maior compromisso: o auno, que é a razão de ser de uma IES!"

\end{enumerate}

\subsubsection{Fonte: Vincent Tinto}
\todo[inline]{soluções apontadas por Vincent Tinto}

\subsubsection{Fonte: Opinião de docentes e de coordenadores acerca do fenômeno da evasão discente dos cursos de graduação da Universidade Federal do Ceará(UFC)}
\todo[inline]{adicionar as sugestões dos docentes e dos coordenadores}


\section{Fonte: site da UFC}

\url{http://ufc.br/a-universidade}

"A Universidade Federal do Ceará é uma autarquia vinculada ao Ministério da Educação. Nasceu como resultado de um amplo movimento de opinião pública. Foi criada pela Lei nº 2.373, em 16 de dezembro de 1954, e instalada em 25 de junho do ano seguinte.

No início, sob a direção de seu fundador, Prof. Antônio Martins Filho, era constituída pela Escola de Agronomia, Faculdade de Direito, Faculdade de Medicina e Faculdade de Farmácia e Odontologia.

Sediada em Fortaleza, Capital do Estado, a UFC é um braço do sistema do Ensino Superior do Ceará e sua atuação tem por base todo o território cearense, de forma a atender às diferentes escalas de exigências da sociedade.

A Universidade é composta de sete campi, denominados Campus do Benfica, Campus do Pici e Campus do Porangabuçu, todos localizados no município de Fortaleza (sede da UFC), além do Campus de Sobral, Campus de Quixadá, Campus de Crateús e Campus de Russas.

A Universidade Federal do Ceará, que há mais de 50 anos mantém o compromisso de servir à região, sem esquecer o caráter universal de sua produção, chega hoje com praticamente todas as áreas do conhecimento representadas em seus campi." 

\section{Fonte: Anuário estatístico 2014 base 2013}

\cite{anuario_2014_base_2013}

\subsection{Objetivos institucionais}

A UFC orienta sua atuação permanentemente no sentido de alcançar os seguintes objetivos:
\begin{enumerate}
\item
Promover a formação humana e profissional de seus estudantes, preparando-os para uma atuação responsável e construtiva na sociedade;
\item
Fomentar a geração de conhecimentos voltados para o desenvolvimento sustentável do Ceará e do Nordeste;
\item
Impulsionar o desenvolvimento, a produção e a preservação da cultura e das artes, com ênfase para as manifestações regionais.
\item
Promover a interação com a sociedade, através da difusão científica, tecnológica, artística e cultural e do desenvolvimento comunitário, sintonizados com as demandas sociais;
\item
Incentivar a capacitação permanente dos quadros docente e técnico-administrativo;
\item
Intensificar e ampliar as relações de parceria e intercâmbio com instituições nacionais e estrangeiras, governamentais e não governamentais;
\item
Buscar a profissionalização da gestão administrativa, apoiada em processos de planejamento e avaliação, executada com base em modelo organizacional flexível, eficiente e eficaz;
\item
Exercitar permanentemente o instituto da autonomia universitária superando restrições e estabelecendo novos parâmetros na gestão e nas relações institucionais;
\item
Assegurar a qualidade no desenvolvimento de todas as ações administrativas e acadêmicas;
\item
Distinguir-se como referência regional pela excelência acadêmica de suas ações nas áreas do ensino, geração do conhecimento e prestação de serviços à população, bem como na produção de arte e cultura.
\end{enumerate}
\textbf{p. 10}

\subsection{Lema}

“O universal pelo regional” é o lema da UFC, instituição que busca centrar seu compromisso na solução dos problemas locais, sem esquecer o caráter universal de sua produção.
\textbf{p. 11}

\subsection{Missão}

A missão da Universidade é formar profissionais da mais alta qualificação, gerar e difundir conhecimentos, preservar e divulgar os valores éticos, científicos, artísticos e culturais, constituindo-se em instituição estratégica para o desenvolvimento do Ceará, do Nordeste e do Brasil.
\textbf{p. 11}

\subsection{Visão}

\underline{Consolidar-se como instituição de referência no ensino de graduação} e pós-graduação (stricto e lato sensu), de preservação, geração e produção de ciência e tecnologia, e de integração com o meio, como forma de contribuir para a superação das desigualdades sociais e econômicas, por meio da promoção do desenvolvimento sustentável do Ceará, do Nordeste e do Brasil.
\textbf{p. 11}

\subsection{Centros, faculdades e institutos}

\begin{enumerate}
\item 
Centro de Ciências
\item 
Centro de Humanidades
\item 
Centro de Tecnologia
\item 
Centro de Ciências Agrárias
\item 
Faculdade de Medicina
\item 
Faculdade de Farmácia, Odontologia e Enfermagem
\item 
Faculdade de Direito
\item 
Faculdade de Educação
\item 
Faculdade de Economia, Administração, Atuariais, Contabilidade e Secretariado Executivo
\item 
Instituto de Ciências do Mar (LABOMAR)
\item 
Instituto de Cultura e Arte (ICA)
\item 
Instituto de Educação Física e Esportes (IEFES)
\item 
Instituto Universidade Virtual – UFC Virtual
\end{enumerate}
\textbf{p. 13}

\subsection{Orçamento da UFC - em milhões}
\begin{tabular}{lrrlr}
\toprule
ESPECIFICAÇÃO &  Previsão &  Executado & Evolução do executado &  Executado/Previsão (\%) \\
\midrule
Ano  &           &            &                       &                         \\
2008 &  804.05 &  771.74 &  - &  95.98 \\
2009 &  748.06 &  774.84 &  0.00\% &  103.58 \\
2010 &  901.89 &  917.41 &  0.18\% &  101.72 \\
2011 &  934.76 &  927.40 &  0.01\% &  99.21 \\
2012 &  974.62 &  977.95 &  0.05\% &  100.34 \\
2013 &  1116.86 &  1119.66 &  0.14\% &  100.25 \\
\bottomrule
\end{tabular}

\textbf{p. 56}

\subsection{Área construída da UFC - mil m2}
\begin{tabular}{lcc}
\toprule
ESPECIFICAÇÃO &  Área Construída (mil m2) & Incremento percentual \\
\midrule
Ano  &                           &                       \\
2003 &  233.63 &  - \\
2004 &  233.63 &  0.00\% \\
2005 &  233.63 &  0.00\% \\
2006 &  235.14 &  0.65\% \\
2007 &  235.14 &  0.00\% \\
2008 &  262.73 &  11.73\% \\
2009 &  277.48 &  5.61\% \\
2010 &  291.31 &  4.98\% \\
2011 &  307.57 &  5.58\% \\
2012 &  317.23 &  3.14\% \\
2013 &  340.67 &  7.39\% \\
\bottomrule
\end{tabular}

\textbf{p. 56}


\subsection{Quantitativos de bolsas por ano e modalidade}

\begin{tabular}{llrrrrrrr}
\toprule
{} &                               ESPECIFICAÇÃO &  2007 &  2008 &  2009 &  2010 &  2011 &  2012 &  2013 \\
\midrule
0  &                    Aprendizagem Cooperativa &     - &     - &    98 &   226 &   250 &   250 &   233 \\
1  &                        Apoio Administrativo &     - &     - &     - &   100 &   100 &   200 &   170 \\
2  &                              Cultura e Arte &     - &     - &    64 &    64 &    64 &    80 &   100 \\
3  &                                    Desporto &     - &     - &     - &    30 &    50 &   100 &   100 \\
4  &                                    Extensão &     - &     - &   274 &   378 &   618 &   670 &   650 \\
5  &                Iniciação Científica - PIBIC &   665 &   726 &   782 &   769 &   942 &   925 &   914 \\
6  &                  Iniciação Acadêmica - PRAE &     - &     - &   500 &   580 &   826 &   756 &   900 \\
7  &   Iniciação à Docência - Remunerada - PIBID &     - &     - &   442 &   605 &   700 &   788 &   732 \\
8  &                                 Informática &     - &     - &    95 &    95 &   100 &   100 &   100 \\
9  &            Monitoria de Projeto (Graduação) &     - &     - &   137 &   256 &   300 &   299 &   276 \\
10 &   Programa de Educação Tutorial - PET - UFC &     - &    24 &    76 &   156 &   280 &   288 &   288 \\
11 &  Programa de Educação Tutorial - PET - SESu &     - &   204 &   204 &   204 &   252 &   252 &   252 \\
\bottomrule
\end{tabular}

\textbf{p. 53}

\subsection{Quantitativos de inscritos(vestibular e sisu) e vagas ofertadas por ano}
\begin{tabular}{lrrr}
\toprule
ESPECIFICAÇÃO &  Candidatos Inscritos &  Vagas Ofertadas &  Demanda \\
\midrule
Ano  &                       &                  &          \\
2005 &  42616 &  3625 &  11.76 \\
2006 &  36719 &  3605 &  10.19 \\
2007 &  37771 &  4045 &  9.34 \\
2008 &  31328 &  4085 &  7.67 \\
2009 &  32490 &  4484 &  7.25 \\
2010 &  44156 &  5524 &  7.99 \\
2011 &  78415 &  5724 &  13.70 \\
2012 &  110914 &  5834 &  19.01 \\
2013 &  133923 &  6308 &  21.23 \\
\bottomrule
\end{tabular}

\textbf{p. 49}

\subsection{Quantitativo de cursos de graduação por ano}
\begin{tabular}{lrrrrrrrrrrr}
\toprule
Ano &  2003 &  2004 &  2005 &  2006 &  2007 &  2008 &  2009 &  2010 &  2011 &  2012 &  2013 \\
\midrule
ESPECIFICAÇÃO &       &       &       &       &       &       &       &       &       &       &       \\
Nº de Cursos  &  53 &  54 &  54 &  69 &  70 &  73 &  76 &  101 &  105 &  108 &  114 \\
\bottomrule
\end{tabular}

\textbf{p. 50}

\subsection{Indicadores de assistência a estudantes}
\begin{tabular}{lrrrr}
\toprule
Ano &  2010 &  2011 &  2012 &  2013 \\
\midrule
ESPECIFICAÇÃO                                         &       &       &       &       \\
Nº de Residências Universitarias                      &  16 &  16 &  14 &  11 \\
Nº de Alunos Residentes (Fortaleza e Interior)        &  416 &  458 &  535 &  622 \\
Nº de Auxílio Residentes (Alunos)                     &  416 &  458 &  535 &  622 \\
Nº de Ajuda de Custo (Alunos)-Participação de eventos &  3648 &  3782 &  3695 &  4273 \\
Nº de Bolsas de Iniciação Acadêmica                   &  580 &  826 &  756 &  900 \\
Nº de Bolsas de Incentivo ao Desporto                 &  30 &  50 &  100 &  100 \\
Nº de Acompanhamento Psicológico                      &  179 &  120 &  190 &  423 \\
\bottomrule
\end{tabular}

\textbf{p. 52}

\subsection{Distribuição de docentes e técnicos-administrativos}

\begin{tabular}{lrrrrrrrrrrr}
\toprule
{} &  2003 &  2004 &  2005 &  2006 &  2007 &  2008 &  2009 &  2010 &  2011 &  2012 &  2013 \\
\midrule
ESPECIFICAÇÃO          &       &       &       &       &       &       &       &       &       &       &       \\
Docentes               &  1083 &  1093 &  1100 &  1234 &  1749 &  1772 &  1956 &  2005 &  2024 &  2052 &  2152 \\
Técnico-administrativo &  3109 &  3453 &  3409 &  3430 &  3366 &  3458 &  3420 &  3408 &  3466 &  3458 &  3407 \\
\bottomrule
\end{tabular}
\textbf{p. 54}

\subsection{Quantitativo de vagas de casa de cultura}

!notar a diminuição gradual de vagas de 2008 a 2013!

\begin{tabular}{lrrrr}
\toprule
ESPECIFICAÇÃO &  Teste de Admissão &  Teste de Seleção &  Matriculados &  Concludentes \\
\midrule
Ano  &                    &                   &                         &                         \\
2003 &  2151 &  1636 &  6197 &  1492 \\
2004 &  1719 &  1194 &  6051 &  1410 \\
2005 &  1670 &  1344 &  5646 &  1308 \\
2006 &  1706 &  1706 &  5796 &  1316 \\
2007 &  1550 &  572 &  5867 &  945 \\
2008 &  1766 &  2363 &  5452 &  1380 \\
2009 &  1691 &  2021 &  5806 &  1447 \\
2010 &  1268 &  1949 &  5579 &  1149 \\
2011 &  1276 &  1803 &  4892 &  1166 \\
2012 &  1078 &  1494 &  4014 &  936 \\
2013 &  1124 &  1407 &  3462 &  1067 \\
\bottomrule
\end{tabular}

\textbf{p. 55}

\subsection{Indicadores de gestão para o TCU}
\cite{indicadores_TCU}

\begin{tabular}{ll}
\toprule
{} &                                                   Indicador \\
\midrule
0  &  AE – Aluno Equivalente da UFC \\
1  &  ATI – Aluno em Tempo Integral \\
2  &  AgE – Aluno Equivalente de Graduação \\
3  &  ApgTI – Aluno da Pós-Graduação em Tempo Integral \\
4  &  ArTI – Aluno de Residência em Tempo Integral \\
5  &  AgTI – Aluno de Graduação em Tempo Integral \\
6  &  Ag – Aluno de Graduação \\
7  &  Apg – Aluno de Pós-Graduação \\
8  &  Ar – Aluno de Residência Médica \\
9  &  Ndi – Alunos Diplomados \\
10 &  Ni – Alunos Ingressantes \\
11 &  Custo corrente com HU (inclui 65\% do HU) \\
12 &  Custo corrente sem HU \\
13 &  Número de funcionários Equivalente com HU \\
14 &  Número de funcionários Equivalente sem HU \\
15 &  Professor Equivalente \\
16 &  I.A. Custo corrente com HU/Aluno Equivalente \\
17 &  I.B. Custo Corrente sem HU/Aluno Equivalente \\
18 &  II. Aluno Tempo Integral/Professor Equivalente \\
19 &  III.A. Aluno Tempo Integral/Funcionário Equivalente com HU \\
20 &  III.B. Aluno Tempo Integral/Funcionário Equivalente sem HU \\
21 &  IV.A. Funcionário Equivalente com HU/Professor Equivalente \\
22 &  IV.B. Funcionário Equivalente sem HU/Professor Equivalente \\
23 &  V. Grau de Participação Estudantil-GPE \\
24 &  VI. Grau de Envolvimento com Pós-Graduação-GEPG \\
25 &  VII. Conceito CAPES para a Pós-Graduação \\
26 &  VIII. Índice de Qualificação do Corpo Docente-IQCD \\
27 &  IX. Taxa de Sucesso na Graduação-TSG \\
\bottomrule
\end{tabular}

\begin{tabular}{lllllll}
\toprule
Ano &          2008 &          2009 &          2010 &          2011 &          2012 &          2013 \\
\midrule
Indic &               &               &               &               &               &               \\
0         &  34023.00 &  33557.62 &  37908.26 &  40708.72 &  41144.35 &  42443.53 \\
1         &  21212.00 &  21461.92 &  23307.93 &  25035.20 &  26330.48 &  26466.34 \\
2         &  28080.00 &  27074.62 &  31631.26 &  33018.72 &  32468.35 &  34247.53 \\
3         &  5615.00 &  6075.00 &  5839.00 &  7308.00 &  8268.00 &  7760.00 \\
4         &  328.00 &  408.00 &  438.00 &  382.00 &  408.00 &  436.00 \\
5         &  15269.00 &  14978.92 &  17030.93 &  17345.20 &  17654.48 &  18270.34 \\
6         &  20991.00 &  21289.00 &  22538.00 &  25971.00 &  26956.00 &  27433.00 \\
7         &  2808.00 &  3038.00 &  2920.00 &  3654.00 &  4134.00 &  3880.00 \\
8         &  164.00 &  204.00 &  219.00 &  191.00 &  204.00 &  218.00 \\
9         &  2520.00 &  2481.00 &  2586.00 &  2792.00 &  2684.00 &  2920.00 \\
10        &  4822.00 &  4731.00 &  6204.00 &  5643.00 &  6406.00 &  6087.00 \\
11        &  444351055.04 &  473411413.49 &  564453156.89 &  581255114.03 &  560737712.22 &  698496687.71 \\
12        &  426930950.49 &  431030343.74 &  513713119.26 &  491835392.86 &  482034252.71 &  609763905.54 \\
13        &  3313.00 &  3252.50 &  3255.50 &  3283.25 &  3281.50 &  3277.75 \\
14        &  1902.25 &  1916.25 &  1954.00 &  1927.00 &  1990.00 &  2047.50 \\
15        &  1619.00 &  1765.50 &  1856.00 &  1851.50 &  1912.50 &  1948.50 \\
16        &  13060.38 &  14107.42 &  14889.98 &  14278.39 &  13628.55 &  16457.08 \\
17        &  12548.36 &  12844.49 &  13551.48 &  12081.82 &  11715.69 &  14366.47 \\
18        &  13.10 &  12.16 &  12.56 &  13.52 &  13.77 &  13.58 \\
19        &  6.40 &  6.60 &  7.16 &  7.63 &  8.03 &  8.07 \\
20        &  11.15 &  11.20 &  11.93 &  12.99 &  13.23 &  12.93 \\
21        &  2.05 &  1.84 &  1.75 &  1.77 &  1.72 &  1.68 \\
22        &  1.17 &  1.09 &  1.05 &  1.04 &  1.04 &  1.05 \\
23        &  0.73 &  0.70 &  0.76 &  0.67 &  0.65 &  0.67 \\
24        &  0.12 &  0.12 &  0.11 &  0.12 &  0.13 &  0.12 \\
25        &  4.13 &  4.11 &  4.22 &  4.22 &  4.20 &  4.34 \\
26        &  3.95 &  3.73 &  4.03 &  4.13 &  4.15 &  4.24 \\
27        &  70.00 &  66.86 &  68.45 &  69.06 &  66.63 &  56.51 \\
\bottomrule
\end{tabular}

\textbf{p. 57~58}

\subsection{Porcentagem de alunos de graduação com bolsas}
\begin{tabular}{llrrrrrr}
\toprule
{} &                                   ESPECIFICAÇÃO &  2008 &  2009 &  2010 &  2011 &  2012 &  2013 \\
\midrule
0 &  Iniciação Científica - PIBIC e PET(Sesu e UFC) &   4.5 &   4.6 &   4.7 &   5.2 &   5.4 &  5.43 \\
1 &                              Bolsa de Monitoria &   5.2 &   5.7 &   7.7 &   7.9 &   7.2 &  6.81 \\
\bottomrule
\end{tabular}

\textbf{p. 53}

\subsection{Bolsas de monitoria, pet - por curso em 2013; de ic - por área de conhecimento 2010 a 2013}
\textbf{p. 282 a 289}

\chapter{Business objectives}

\begin{itemize}
\item Redução de desperdício financeiro - \cite{evasao_global}
\item Redução de abandonos efetivos subsequentes a abandonos temporários
\item REUNI \cite{reuni}
\item PDI \cite{pdi_ufc}
\item Programa de gestão da atual reitoria \cite{henry}
\end{itemize}

\chapter{Business success criteria}

\cite{anuario_2014_base_2013}
\cite{pdi_ufc}

\bibliographystyle{acm}
\bibliography{../../../referencias}

\end{document}
